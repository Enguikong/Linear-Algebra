\documentclass{article}
\usepackage{xeCJK,amsmath,geometry,graphicx,amssymb,zhnumber,booktabs,setspace,tasks,verbatim,amsthm,amsfonts,mathdots}
\geometry{a4paper,scale=0.8}   
\title{线性代数homework (第十一周)}
\author{PB20000113孔浩宇}
\date{\today}
\begin{document}
\maketitle
\section{周二}
\subsection{习题六}
\begin{enumerate}
    \item [8.]设$\mathcal{A}$在基$\alpha_1,\alpha_2,\alpha_3$下的矩阵为
    \[
        A=\begin{pmatrix}
            1 & 2 & 3\\
            -1 & 0 & 3\\
            2 & 1 & 5
        \end{pmatrix},
    \]
    求$\mathcal{A}$在基$\beta_1,\beta_2,\beta_3$下的矩阵:
    \begin{enumerate}
        \item [(1)]$\beta_1=\alpha_3,\ \beta_2=\alpha_1,\ \beta_3=\alpha_2$;
        \item [(2)]$\beta_1=\alpha_1,\beta_2=\alpha_2+\alpha_3,\beta_3=\alpha_2-\alpha_3$.
    \end{enumerate}
    \begin{enumerate}
        \item [(1)]设$\mathcal{A}$在基$\beta_1,\beta_2,\beta_3$下的矩阵为$B$.
        \[
            (\beta_1,\beta_2,\beta_3)=(\alpha_1,\alpha_2,\alpha_3)
            \begin{pmatrix}
                0 & 1 & 0\\
                0 & 0 & 1\\
                1 & 0 & 0
            \end{pmatrix}
        \]
        \[
            \Rightarrow
            B=
            {\begin{pmatrix}
                0 & 1 & 0\\
                0 & 0 & 1\\
                1 & 0 & 0
            \end{pmatrix}}^{-1}
            \begin{pmatrix}
                1 & 2 & 3\\
                -1 & 0 & 3\\
                2 & 1 & 5
            \end{pmatrix}
            \begin{pmatrix}
                0 & 1 & 0\\
                0 & 0 & 1\\
                1 & 0 & 0
            \end{pmatrix}
            =
            \begin{pmatrix}
                5 & 2 & 1\\
                3 & 1 & 2\\
                3 & -1 & 0
            \end{pmatrix}.
        \]
        \item [(2)]设$\mathcal{A}$在基$\beta_1,\beta_2,\beta_3$下的矩阵为$B$.
        \[
            (\beta_1,\beta_2,\beta_3)=(\alpha_1,\alpha_2,\alpha_3)
            \begin{pmatrix}
                1 & 0 & 0\\
                0 & 1 & 1\\
                0 & 1 & -1
            \end{pmatrix}
        \]
        \[
            \Rightarrow
            B=
            {\begin{pmatrix}
                1 & 0 & 0\\
                0 & 1 & 1\\
                0 & 1 & -1
            \end{pmatrix}}^{-1}
            \begin{pmatrix}
                1 & 2 & 3\\
                -1 & 0 & 3\\
                2 & 1 & 5
            \end{pmatrix}
            \begin{pmatrix}
                1 & 0 & 0\\
                0 & 1 & 1\\
                0 & 1 & -1
            \end{pmatrix}
            =\begin{pmatrix}
                1 & 5 & -1\\
                & &\\
                \displaystyle{\frac{1}{2}} & \displaystyle{\frac{9}{2}} & -\displaystyle{\frac{7}{2}}\\
                & &\\
                -\displaystyle{\frac{3}{2}} & -\displaystyle{\frac{3}{2}} & \displaystyle{\frac{1}{2}}
            \end{pmatrix}.
        \]
    \end{enumerate}
    \item [10.]如果$A$与$B$相似,$C$与$D$相似,证明
    $\begin{pmatrix}
        A & O\\
        O & C
    \end{pmatrix}$与
    $\begin{pmatrix}
        B & O\\
        O & D
    \end{pmatrix}$相似.
    \begin{proof}
        设有可逆方阵$P$与$Q$满足$P^{-1} A P=B,\ Q^{-1} C Q=D$,则
        \[
            \begin{pmatrix}
                P^{-1} & O\\
                O & Q^{-1} 
            \end{pmatrix}
            \begin{pmatrix}
                A & O\\
                O & C
            \end{pmatrix}
            \begin{pmatrix}
                P & O\\
                O & Q
            \end{pmatrix}
            =\begin{pmatrix}
                P^{-1} A P & O\\
                O & Q^{-1} C Q
            \end{pmatrix}
            =\begin{pmatrix}
                B & O\\
                O & D
            \end{pmatrix}.
        \]
        又
        \[
            \begin{pmatrix}
                P^{-1} & O\\
                O & Q^{-1} 
            \end{pmatrix}
            \begin{pmatrix}
                P & O\\
                O & Q
            \end{pmatrix}
            =I.
        \]
        即证.
    \end{proof}
    \item [11.]设方阵$A$与$B$相似,证明:
    \begin{enumerate}
        \item [(1)]对每个整数$k$,$A^k$相似于$B^k$;
        \item [(2)]对每个多项式$f$,$f(A)$相似于$f(B)$.
    \end{enumerate}
    \begin{proof}
        \begin{enumerate}
            \item []设有可逆方阵$P,P^{-1} A P=B$.
            \item [(1)]
            \[ 
                P^{-1} A^{k} P={(P^{-1}A P)}^{k}=B^{k}.
            \]
            \item [(2)]设$f(x)=a_0+a_1 x+\cdots+a_n x^n$.
            \begin{align*}
                P^{-1} f(A) P
                =&P^{-1} (a_0 I+a_1 A +\cdots +a_n A^n) P\\
                =&a_0 I+a_1 P^{-1} A P+\cdots +a_n P^{-1} A^n P\\
                =& a_0 I+a_1 B +\cdots +a_n B^n.
            \end{align*}
        \end{enumerate}
    \end{proof}
\end{enumerate}
\section{周四}
\subsection{习题六}
\end{document}