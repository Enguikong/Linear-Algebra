\documentclass{article}
\usepackage{xeCJK,amsmath,geometry,graphicx,amssymb,zhnumber,booktabs,setspace,tasks,verbatim,amsthm,amsfonts,mathdots,mathtools}
\geometry{a4paper,scale=0.8}
\title{线性代数homework (第五周)}
\author{PB20000113孔浩宇}
\date{\today}
\begin{document}
\maketitle
\section{周二}
\subsection{习题四}
    \begin{enumerate}
        \item [36.]
        \begin{enumerate}
            \item [(1)]做初等变换
            \begin{align*}
                &\begin{pmatrix}
                    3  & 2 & -1 & 9\\
                    -2 & 1 & -4 & 2\\
                    -1 & -2& 3  & -2\\
                    3  & 2 & -1 & 9
                \end{pmatrix}
                \longrightarrow 
                \begin{pmatrix}
                    0 & -4 & -1 & 9\\
                    0 & 5  & -10& 6\\
                    -1& -2 & 3 & -2\\
                    0 & 0 & 0 & 0
                \end{pmatrix}
                \longrightarrow
                \begin{pmatrix}
                    1 & 0 & 0 & 0\\
                    0 & -4 & -1 & 9\\
                    0 & 5 & -10 & 6\\
                    0 & 0 & 0 & 0
                \end{pmatrix}\\
                \longrightarrow
                &\begin{pmatrix}
                    1 & 0 & 0 & 0\\
                    0 & 0 & -1 & 0\\
                    0 & 35 & 0 & -84\\
                    0 & 0 & 0 & 0
                \end{pmatrix}
                \longrightarrow
                \begin{pmatrix}
                    1 & 0 & 0 & 0\\
                    0 & 0 & -1 & 0\\
                    0 & 0 & 0 & -84\\
                    0 & 0 & 0 & 0
                \end{pmatrix}
                \longrightarrow
                \begin{pmatrix}
                    1 & 0 & 0 & 0\\
                    0 & 1 & 0 & 0\\
                    0 & 0 & 1 & 0\\
                    0 & 0 & 0 & 0
                \end{pmatrix}
            \end{align*}
            即\[
                \mbox{rank}\begin{pmatrix}
                    3  & 2 & -1 & 9\\
                    -2 & 1 & -4 & 2\\
                    -1 & -2& 3  & -2\\
                    3  & 2 & -1 & 9
                \end{pmatrix}=3.
            \]
        \end{enumerate}
        \item [37.]对于$a,b$的各种取值,讨论实矩阵
        $\begin{pmatrix}
            1 & 2 & 3\\
            2 & 4 & a\\
            3 & b & 9
        \end{pmatrix}$的秩.
        \[
            \begin{pmatrix}
                1 & 2 & 3\\
                2 & 4 & a\\
                3 & b & 9
            \end{pmatrix}
            \xrightarrow[-3r_1 \to r_3]{-2r_1 \to r_2}
            \begin{pmatrix}
                1 & 2 & 3\\
                0 & 0 & a-6\\
                0 & b-6 & 0
            \end{pmatrix}
            \xrightarrow[-3c_1 \to c_3]{-2c_1 \to c_2}
            \begin{pmatrix}
                1 & 0 & 0\\
                0 & 0 & a-6\\
                0 & b-6 & 0
            \end{pmatrix}
        \]
        即
        \[
            \mbox{rank}\begin{pmatrix}
                1 & 2 & 3\\
                2 & 4 & a\\
                3 & b & 9
            \end{pmatrix}=
            1+\mbox{rank}\begin{pmatrix}
                0 & a-6\\
                b-6 & 0
            \end{pmatrix}.
        \]
        \begin{enumerate}
            \item [(1)]$a-6=0,b-6=0$
            \[
                \mbox{rank}\begin{pmatrix}
                    1 & 2 & 3\\
                    2 & 4 & a\\
                    3 & b & 9
                \end{pmatrix}=
                1+\mbox{rank}\begin{pmatrix}
                    0 & 0\\
                    0 & 0
                \end{pmatrix}=1.\quad (a=6,b=6)
            \]
            \item [(2)]$a-6=0,b-0\neq 0$
            \[
                \mbox{rank}\begin{pmatrix}
                    1 & 2 & 3\\
                    2 & 4 & a\\
                    3 & b & 9
                \end{pmatrix}=
                1+\mbox{rank}\begin{pmatrix}
                    0 & 0\\
                    b-6 & 0
                \end{pmatrix}=2.\quad (a=6,b\neq 6)
            \]
            \item [(3)]$a-6\neq 0,b-6=0$
            \[
                \mbox{rank}\begin{pmatrix}
                    1 & 2 & 3\\
                    2 & 4 & a\\
                    3 & b & 9
                \end{pmatrix}=
                1+\mbox{rank}\begin{pmatrix}
                    0 & a-6\\
                    0 & 0
                \end{pmatrix}=2.\quad (a\neq 6,b=6)
            \]
            \item [(4)]$a-6\neq 0,b-6\neq 0$
            \[
                \mbox{rank}\begin{pmatrix}
                    1 & 2 & 3\\
                    2 & 4 & a\\
                    3 & b & 9
                \end{pmatrix}=
                1+\mbox{rank}\begin{pmatrix}
                    0 & a-6\\
                    b-6 & 0
                \end{pmatrix}=3.\quad (a\neq 6,b\neq 6)
            \]
        \end{enumerate}
        \item [39.]设$A$是n阶矩阵,证明:$\mbox{rank}(A^*)=\begin{cases}
            n, & \mbox{rank}(A)=n,\\
            1, & \mbox{rank}(A)=n-1,\\
            0, & \mbox{rank}(A)\leqslant n-2.
        \end{cases}$
        \begin{proof}
            \begin{enumerate}
                \item []
                \item [(0)]
                引理1:$\quad \mbox{rank}\begin{pmatrix} A & O\\ C & B \end{pmatrix} \geqslant \mbox{rank}(A)+\mbox{rank}(B)$.
                \begin{proof}
                    记$a=\mbox{rank}(A),b=\mbox{rank}(B)$,则存在可逆方阵$P_1,Q_1,P_2,Q_2$使
                \[
                    P_1 A Q_1=\begin{pmatrix}
                        I_a & O\\
                        O & O
                    \end{pmatrix},\quad
                    P_2 B Q_2=\begin{pmatrix}
                        I_b & O\\
                        O & O
                    \end{pmatrix}
                \]
                取可逆方阵
                \[
                    P=\begin{pmatrix}
                        P_1 & O\\
                        O & P_2
                    \end{pmatrix},\quad
                    Q=\begin{pmatrix}
                        Q_1 & O\\
                        O & Q_2
                    \end{pmatrix}
                \]
                则
                \[
                    S=P\begin{pmatrix}
                        A & O\\
                        C & B
                    \end{pmatrix}Q
                    =\begin{pmatrix}
                        P_1 A Q_1 & O\\
                        P_2 C Q_1 & P_2 B Q_2
                    \end{pmatrix}
                    =\begin{pmatrix}
                        diag(I_a,O) & O\\
                        P_2 C Q_2 & diag(I_b,O)
                    \end{pmatrix}
                \]
                存在$a+b$阶子式$\begin{vmatrix} I_a & O\\ * & I_b \end{vmatrix}=1\neq 0$,因此
                \[
                    \mbox{rank}\begin{pmatrix}
                        A & O\\
                        C & B
                    \end{pmatrix}
                    =\mbox{rank}(S)\geqslant a+b =\mbox{rank}(A)+\mbox{rank}(B).
                \]
                \end{proof}
                引理2:$\quad \mbox{rank}(A)+\mbox{rank}B-n\leqslant \mbox{rank}(AB)$. (Sylvecter秩不等式)
                \begin{proof}
                即\[\mbox{rank}(A)+\mbox{rank}(B)\leqslant \mbox{rank}(I_n)+\mbox{rank}(AB)=\mbox{rank}\begin{pmatrix}AB & O \\ O & I_n \end{pmatrix}\]
                进行初等变换
                \[
                    \begin{pmatrix}
                        I_n & A\\
                        O & I_n
                    \end{pmatrix}
                    \begin{pmatrix}
                        AB & O\\
                        O & I_n
                    \end{pmatrix}
                    \begin{pmatrix}
                        I_n & O\\
                        -B & I_n
                    \end{pmatrix}
                    =
                    \begin{pmatrix}
                        O & A\\
                        -B & I_n
                    \end{pmatrix},
                    \quad
                    \begin{pmatrix}
                        O & A\\
                        -B & I_n
                    \end{pmatrix}
                    \begin{pmatrix}
                        O & -I_n\\
                        I_n & O
                    \end{pmatrix}
                    =
                    \begin{pmatrix}
                        A & O\\
                        I_n & B
                    \end{pmatrix}
                \]
                即有
                \[
                    \mbox{rank}\begin{pmatrix}
                        AB & O\\
                        O & I_n
                    \end{pmatrix}=
                    \mbox{rank}\begin{pmatrix}
                        A & O\\
                        I_n & B
                    \end{pmatrix}
                    \geqslant \mbox{rank}(A)+\mbox{rank}(B).
                \]
                \end{proof}
                \item [(1)]$\mbox{rank}(A)=n$
                \[\mbox{rank}(A^*)=\mbox{rank}(A\cdot A^*)=\mbox{rank}(\det(A)I_n)=n.\]
                \item [(2)]$\mbox{rank}(A)=n-1$
                \[
                    \begin{cases}
                        \ A \cdot A^*=O &\Rightarrow \mbox{rank}(A^*)+\mbox{rank}(A)\leqslant n\\
                        \ &\\
                        \ \mbox{rank}(A)=n-1 &\Rightarrow \exists\ A_{ij}\neq 0 , \mbox{rank}(A^*)>0
                    \end{cases}
                    \Rightarrow
                    0< \mbox{rank}(A^*) \leqslant n-\mbox{rank}(A)=1.
                \]
                显然有\[\mbox{rank}(A^*)=1.\]
                \item [(3)]$\mbox{rank}(A)\leqslant n-2$
                \[\mbox{rank}(A) \leqslant n-2 \Rightarrow \forall\ i,j,A_{ij}={(-1)}^{i+j} M_{ij}=0\Rightarrow \mbox{rank}(A^*)=\mbox{rank}(O)=0.\]
            \end{enumerate}
        \end{proof}
        \item [41.]证明下列关于秩的等式和不等式: (其中$A,B,C$是使运算有意义的矩阵)
        \begin{enumerate}
            \item [(1)]$\mbox{max}\left( \mbox{rank}(A),\mbox{rank}(B),\mbox{rank}(A+B)\right)\leqslant \mbox{rank} \begin{pmatrix} A & B\end{pmatrix}$;
            \item [(2)]$\mbox{rank} \begin{pmatrix} A & B\end{pmatrix}\leqslant \mbox{rank} (A) +\mbox{rank} (B)$;
            \item [(3)]$\mbox{rank} \begin{pmatrix} A & C\\ O & B\end{pmatrix}\geqslant \mbox{rank}(A)+\mbox{rank} (B)$.
        \end{enumerate}
        \begin{proof}
            \begin{enumerate}
                \item []
                \item [(1)]任意$A,B$的非零子式均为$\begin{pmatrix} A & B\end{pmatrix}$的非零子式,
                即\[\mbox{rank}\begin{pmatrix} A & B\end{pmatrix}\geqslant \mbox{rank}(A),
                \quad \mbox{rank}\begin{pmatrix} A & B\end{pmatrix}\geqslant \mbox{rank}(B)\]
                记$A=\begin{pmatrix}\boldsymbol{a}_1 & \boldsymbol{a}_2 & \cdots & \boldsymbol{a}_n \end{pmatrix},
                B=\begin{pmatrix}\boldsymbol{b}_1 & \boldsymbol{b}_2 & \cdots & \boldsymbol{b}_n \end{pmatrix}$.
                则\[
                    \mbox{rank}\begin{pmatrix} A & B\end{pmatrix}=
                    \mbox{rank}\left\{\boldsymbol{a}_1,\ldots,\boldsymbol{a}_n,\boldsymbol{b}_1,\ldots,\boldsymbol{b}_n \right\}
                    \geqslant
                    \mbox{rank}\left\{\boldsymbol{a}_1+\boldsymbol{b}_1,\ldots,\boldsymbol{a}_n+\boldsymbol{b}_n \right\}
                    =\mbox{rank}(A+B).
                \]
                综上\[\mbox{max}\left( \mbox{rank}(A),\mbox{rank}(B),\mbox{rank}(A+B)\right)\leqslant \mbox{rank} \begin{pmatrix} A & B\end{pmatrix}.\]
                \item [(2)]记$a=\mbox{rank}(A),b=\mbox{rank}(B),
                A=\begin{pmatrix}
                    \boldsymbol{a}_1 & \boldsymbol{a}_2 & \cdots & \boldsymbol{a}_m
                \end{pmatrix},
                B=\begin{pmatrix}
                    \boldsymbol{b}_1 & \boldsymbol{b}_2 & \cdots & \boldsymbol{b}_n
                \end{pmatrix}$.
                \[
                    \forall \left\{\boldsymbol{a}_{i1} , \boldsymbol{a}_{i2} , \ldots , \boldsymbol{a}_{ia}\right\}
                    \subseteq  \left\{\boldsymbol{a}_1 , \boldsymbol{a}_2 , \ldots , \boldsymbol{a}_m\right\},\ 
                    \exists \ x_1,\ldots,x_a,x_1\cdot\boldsymbol{a}_{i1}+\cdots+x_a\cdot\boldsymbol{a}_{ia}=0.
                \]
                \[
                    \forall \left\{\boldsymbol{b}_{j1} , \boldsymbol{b}_{j2} , \ldots , \boldsymbol{b}_{jb}\right\}
                    \subseteq  \left\{\boldsymbol{b}_1 , \boldsymbol{b}_2 , \ldots , \boldsymbol{b}_n\right\},\ 
                    \exists \ y_1,\ldots,y_b,y_1\cdot\boldsymbol{b}_{j1}+\cdots+y_b\cdot\boldsymbol{b}_{jb}=0.
                \]
                $\forall \left\{\boldsymbol{a}_{i1},\ldots,\boldsymbol{a}_{ia},\boldsymbol{b}_{j1},\ldots,\boldsymbol{b}_{jb}\right\}
                    \subseteq  \left\{\boldsymbol{a}_1,\ldots,\boldsymbol{a}_m,\boldsymbol{b}_1,\ldots,\boldsymbol{b}_n\right\}$
                \[
                    \exists \ x_1,\ldots,x_a,y_1,\ldots,y_b,
                    x_1\cdot\boldsymbol{a}_{i1}+\cdots+x_a\cdot\boldsymbol{a}_{ia}
                    +y_1\cdot\boldsymbol{b}_{j1}+\cdots+y_b\cdot\boldsymbol{b}_{jb}=0.
                \]
                即
                \[
                    \mbox{rank}
                    \begin{pmatrix}
                        A & B
                    \end{pmatrix}=
                    \mbox{rank}
                    \left\{\boldsymbol{a}_1,\ldots,\boldsymbol{a}_m,\boldsymbol{b}_1,\ldots,\boldsymbol{b}_n\right\}
                    \leqslant a+b=\mbox{rank}(A)+\mbox{rank}(B).
                \]
                \item [(3)]记$a=\mbox{rank}(A),b=\mbox{rank}(B)$,则存在可逆方阵$P_1,Q_1,P_2,Q_2$使
                \[
                    P_1 A Q_1=\begin{pmatrix}
                        I_a & O\\
                        O & O
                    \end{pmatrix},\quad
                    P_2 B Q_2=\begin{pmatrix}
                        I_b & O\\
                        O & O
                    \end{pmatrix}
                \]
                取可逆方阵
                \[
                    P=\begin{pmatrix}
                        P_1 & O\\
                        O & P_2
                    \end{pmatrix},\quad
                    Q=\begin{pmatrix}
                        Q_1 & O\\
                        O & Q_2
                    \end{pmatrix}
                \]
                则
                \[
                    S=P\begin{pmatrix}
                        A & C\\
                        O & B
                    \end{pmatrix}Q
                    =\begin{pmatrix}
                        P_1 A Q_1 & P_1 C Q_2\\
                        O & P_2 B Q_2
                    \end{pmatrix}
                    =\begin{pmatrix}
                        diag(I_a,O) & P_1 C Q_2\\
                        O & diag(I_b,O)
                    \end{pmatrix}
                \]
                存在$a+b$阶子式$\begin{vmatrix} I_a & *\\ O & I_b \end{vmatrix}=1\neq 0$,因此
                \[
                    \mbox{rank}\begin{pmatrix}
                        A & C\\
                        O & B
                    \end{pmatrix}
                    =\mbox{rank}(S)\geqslant a+b =\mbox{rank}(A)+\mbox{rank}(B).
                \]
            \end{enumerate}
        \end{proof}
        \item [43.]设$n$阶方阵$A$满足$A^2=I$,证明:$\mbox{rank}(I+A)+\mbox{rank}(I-A)=n$.
        \begin{proof}
            进行初等变换
            \[
                \begin{pmatrix}
                    I+A & O\\
                    O & I-A
                \end{pmatrix}
                \longrightarrow
                \begin{pmatrix}
                    I+A & I-A\\
                    O & I-A
                \end{pmatrix}
                \longrightarrow
                \begin{pmatrix}
                    I+A & 2I\\
                    O & I-A
                \end{pmatrix}
            \]
            \[
                \longrightarrow
                \begin{pmatrix}
                    O & 2I\\
                    -\frac{1}{2}(I-A)(I+A) & I-A
                \end{pmatrix}
                \longrightarrow
                \begin{pmatrix}
                    O & 2I\\
                    O & I-A
                \end{pmatrix}
                \longrightarrow
                \begin{pmatrix}
                    O & 2I\\
                    O & O
                \end{pmatrix}
            \]
            即
            \[
                \mbox{rank}(I+A)+\mbox{rank}(I-A)=
                \mbox{rank}\begin{pmatrix}
                    I+A & O\\
                    O & I-A
                \end{pmatrix}=
                \mbox{rank}\begin{pmatrix}
                    O & 2I\\
                    O & O
                \end{pmatrix}
                =n.
            \]
        \end{proof}
    \end{enumerate}
\section{周四}
\subsection{习题五}
    \begin{enumerate}
        \item [3.]在$\mathbb{F}^4$中,判断向量$\boldsymbol{b}$能否写成$\boldsymbol{a}_1,\boldsymbol{a}_2,\boldsymbol{a}_3$的线性组合.若能,请写出一种表示方式.
        \begin{enumerate}
            \item [(1)]$\boldsymbol{a}_1=(-1,3,0,-5),\quad \boldsymbol{a}_2=(2,0,7,-3)$,
            \item []$\boldsymbol{a}_3=(-4,1,-2,6),\quad \boldsymbol{b}=(8,3,-1,-25)$;
            \item [(2)]$\boldsymbol{a}_1={(3,-5,2,-4)}^T,\quad \boldsymbol{a}_2={(-1,7,-3,6)}^T$,
            \item []$\boldsymbol{a}_3={(3,11,-5,10)}^T,\quad \boldsymbol{b}={(2,-30,13,-26)}^T$;
        \end{enumerate}
        假设存在$x,y,z\in \mathbb{R},\boldsymbol{b}=x\cdot\boldsymbol{a}_1+y\cdot\boldsymbol{a}_2+z\cdot\boldsymbol{a}_3$.
        \begin{enumerate}
            \item [(1)]
            \[
                \boldsymbol{b}=x\cdot\boldsymbol{a}_1+y\cdot\boldsymbol{a}_2+z\cdot\boldsymbol{a}_3
                \Leftrightarrow
                \begin{cases}
                    \ -\ x+2y-4z &=8,\\
                    \ \ \;3x\qquad\ +\ \,z &=3,\\
                    \ \qquad \ \ \ 7y-2z &= -1,\\
                    \ -5x-3y+6z &=-25.
                \end{cases}
                \Leftrightarrow
                \begin{cases}
                    x&=2,\\
                    y&=-1,\\
                    z&=-3.
                \end{cases}
            \]
            能,$\quad\boldsymbol{b}=2\boldsymbol{a}_1-\boldsymbol{a}_2-3\boldsymbol{a}_3$.
            \item [(2)]
            \[
                \boldsymbol{b}=x\cdot\boldsymbol{a}_1+y\cdot\boldsymbol{a}_2+z\cdot\boldsymbol{a}_3
                \Leftrightarrow
                \begin{cases}
                    \ \ \ \,3x-\ \,y+\ 3z&=2,\\
                    \ -5x+7y+11z&=-30,\\
                    \ \ \ \,2x-3y-\ 5z&=13,\\
                    \ -4x+6y+10z&=-26.
                \end{cases}
                \Leftrightarrow
                \begin{cases}
                    x&=-1,\\
                    y&=-5,\\
                    z&=0.
                \end{cases}
            \]
            能,$\quad\boldsymbol{b}=-\boldsymbol{a}_1-5\boldsymbol{a}_2$.
        \end{enumerate}
        \item [4.]设$\boldsymbol{a}_1=(1,0,0,0),\boldsymbol{a}_2=(1,1,0,0),\boldsymbol{a}_3=(1,1,1,0),\boldsymbol{a}_4=(1,1,1,1)$.
        证明:$\mathbb{F}^4$中任何向量都可以写成 的线性组合,且表示唯一.
        \begin{proof}
            \[
            \forall\ \vec{\nu}\in\mathbb{F}^4,\ \exists!(x_1,x_2,x_3,x_4),
            \vec{\nu}=x_1 \boldsymbol{e}_1+x_2 \boldsymbol{e}_2+x_3 \boldsymbol{e}_3+x_4 \boldsymbol{e}_4.
            \]
            \[
                \mbox{又}
                \begin{cases}
                    \boldsymbol{e}_1&=\boldsymbol{a}_1,\\
                    \boldsymbol{e}_2&=\boldsymbol{a}_2-\boldsymbol{a}_1,\\
                    \boldsymbol{e}_3&=\boldsymbol{a}_3-\boldsymbol{a}_2,\\
                    \boldsymbol{e}_4&=\boldsymbol{a}_4-\boldsymbol{a}_3.
                \end{cases}
                \Rightarrow
                \ \exists!(x_1,x_2,x_3,x_4),
                \vec{\nu}=x_1 \boldsymbol{a}_1+x_2 \boldsymbol{a}_2+x_3 \boldsymbol{a}_3+x_4 \boldsymbol{a}_4.
            \]
        \end{proof}
        \item [5.]设$P_i=(x_i,y_i,z_i),i=1,2,3,4$是三维几何空间中的点.证明:$P_i,i=1,2,3,4$共面的条件是
        \[
            \begin{vmatrix}
                x_1 & y_1 & z_1 & 1\\
                x_2 & y_2 & z_2 & 1\\
                x_3 & y_3 & z_3 & 1\\
                x_4 & y_4 & z_4 & 1
            \end{vmatrix}=0.
        \]
        \begin{proof}
            记$\alpha_1=\overrightarrow{P_1 P_4},\alpha_2=\overrightarrow{P_2 P_4},\alpha_3=\overrightarrow{P_3 P_4}$.
            \[
                \mbox{四点共面}
                \Leftrightarrow
                \alpha_1,\alpha_2,\alpha_3 \mbox{共面}
                \Leftrightarrow
                \exists\ m,n\in \mathbb{R},m\alpha_1+n\alpha_2=\alpha_3,
                \begin{pmatrix}
                    x_3-x_4\\
                    y_3-y_4\\
                    z_3-y_4
                \end{pmatrix}
                =m\begin{pmatrix}
                    x_1-x_4\\
                    y_1-y_4\\
                    z_1-y_4
                \end{pmatrix}
                +n\begin{pmatrix}
                    x_2-x_4\\
                    y_2-y_4\\
                    z_2-y_4
                \end{pmatrix}.
            \]
            即
            \[
                P_i,i=1,2,3,4\mbox{共面}
                \Leftrightarrow
                \begin{pmatrix}
                    x_3-x_4\\
                    y_3-y_4\\
                    z_3-y_4
                \end{pmatrix}
                ,\begin{pmatrix}
                    x_1-x_4\\
                    y_1-y_4\\
                    z_1-y_4
                \end{pmatrix}
                ,\begin{pmatrix}
                    x_2-x_4\\
                    y_2-y_4\\
                    z_2-y_4
                \end{pmatrix}
                \mbox{线性相关}
                \Leftrightarrow
                \begin{vmatrix}
                    x_1-x_4 & x_2-x_4 & x_3-x_4\\
                    y_1-y_4 & y_2-y_4 & y_3-y_4\\
                    z_1-z_4 & z_2-z_4 & z_3-z_4
                \end{vmatrix}=0.
            \]
            \[
                \begin{vmatrix}
                    x_1 & y_1 & z_1 & 1\\
                    x_2 & y_2 & z_2 & 1\\
                    x_3 & y_3 & z_3 & 1\\
                    x_4 & y_4 & z_4 & 1
                \end{vmatrix}=
                \begin{vmatrix}
                    x_1-x_4 & y_1-y_4 & z_1-z_4 & 0\\
                    x_2-x_4 & y_2-y_4 & z_2-z_4 & 0\\
                    x_3-x_4 & y_3-y_4 & z_3-z_4 & 0\\
                    x_4 & y_4 & z_4 & 1
                \end{vmatrix}
                =
                \begin{vmatrix}
                    x_1-x_4 & x_2-x_4 & x_3-x_4\\
                    y_1-y_4 & y_2-y_4 & y_3-y_4\\
                    z_1-z_4 & z_2-z_4 & z_3-z_4
                \end{vmatrix}=0.
                \Leftrightarrow
                P_i,i=1,2,3,4\mbox{共面}.
            \]
        \end{proof}
        \item [6.]设$\boldsymbol{a}_1,\boldsymbol{a}_2,\boldsymbol{a}_3,\boldsymbol{a}_4$是三维几何空间中的四个向量.证明它们必线性相关.
    \end{enumerate}
\end{document} 