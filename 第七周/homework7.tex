\documentclass{article}
\usepackage{xeCJK,amsmath,geometry,graphicx,amssymb,zhnumber,booktabs,setspace,tasks,verbatim,amsthm,amsfonts,mathdots,mathtools}
\geometry{a4paper,scale=0.8}
\MHInternalSyntaxOn
\def\MT_start_cases_ams:n #1{%
  \RIfM@\else
   \nonmatherr@{\begin{\@currenvir}}
  \fi
  \MH_group_align_safe_begin:
  \left#1
  \alignedat@a
}
\def\MH_end_cases_ams:{%
  \endaligned
  \MH_group_align_safe_end:
}
\newcommand*\defcases[3]{%
 \newenvironment{#1}
   {\MT_start_cases_ams:n {#2}}
   {\MH_end_cases_ams:\right#3}
}
\MHInternalSyntaxOff
\defcases{mylcases}{\lbrace}{.}
\defcases{myrcases}{.}{\rbrace}
\defcases{mybcases}{\lbrace}{\rbrace}
\defcases{mylrcases}{[}{\rbrace}
\title{线性代数homework (第七周)}
\author{PB20000113孔浩宇}
\date{\today}
\begin{document}
\maketitle
\section{周四}
\subsection{习题五}
\begin{enumerate}
    \item [9.]判别下列线性方程组是否线性相关:
    \[
        (1)\begin{mylcases}{7}
            \ -x_1\; & +\; & 2 & x_2\; & +\; & 3  & x_3 &=3,\\
            \ 5x_1\; &  \; &   &    \; & -\; &    & x_3 &=5,\\
            \ 8x_1\; & -\; & 6 & x_2\; & -\; & 10 & x_3 &=7;
        \end{mylcases}\qquad\qquad
        (2)\begin{mylcases}{10}
            \ 2 x_1\; & +\; & 3 & x_2\; & +\; &   & x_3\; & +\; & 5 & x_4 & = 2,\\
            \ 3 x_1\; & +\; & 2 & x_2\; & +\; & 4 & x_3\; & +\; & 2 & x_4 & = 3,\\
            \   x_1\; & +\; &   & x_2\; & +\; & 2 & x_3\; & +\; & 4 & x_4 & = 1.
        \end{mylcases}
    \]
    \begin{enumerate}
        \item [(1)]
        令$\begin{cases}
            l_1 &=-x_1+2x_2+3x_3-4,\\
            l_2 &=5x_1-x_3+1,\\
            l_3 &=8x_1-6x_2-10x_3+13.
        \end{cases}$,假设$l_3=\lambda_1 l_1+\lambda_2 l_2$,则
        \[
            \begin{cases}
                -\ \lambda_1+5\lambda_2 &=8,\\
                \ \ 2\lambda_1 &=-6,\\
                \ \ 3\lambda_1-\ \lambda_2 &=-10,\\
                -4\lambda_1+\ \lambda_2 &=13.
            \end{cases}
            \quad \Rightarrow
            \begin{cases}
                \lambda_1 &=-3,\\
                \lambda_2 &=1.
            \end{cases}
            \quad \Rightarrow
            l_3=-3l_1+l_2,\mbox{原方程组线性相关.}
        \]
        \item [(2)]令$\begin{cases}
            \ l_1 &=2x_1+3x_2+x_3+5x_4-2,\\
            \ l_2 &=3x_1+2x_2+4x_3+2x_4-3,\\
            \ l_3 &=x_1+x_2+2x_3+4x_4-1.
        \end{cases}$,假设$l_3=\lambda_1 l_1+\lambda_2 l_2$,则
        \[
            \begin{cases}
                \ 2\lambda_1+3\lambda_2 &=1 ,\\
                \ 3\lambda_1+2\lambda_2 &=1 ,\\
                \ \lambda_1+4\lambda_2 &=2 ,\\
                \ 5\lambda_1+2\lambda_2 &=4 ,\\
                \ -2\lambda_1-3\lambda_2 &= -1,\\
            \end{cases}
            \quad \Rightarrow
            \mbox{无解,即原方程组不线性相关.}
        \]
    \end{enumerate}
    \item [10.]判断下列向量组是否线性相关:
    \begin{enumerate}
        \item [(2)]$a_1=(2,1,2,-4),a_2=(1,0,5,2),a_3=(-1,2,0,3)$.
        
        假设有$a_3=m\cdot a_1+n\cdot a_2$,则
        \[
            \begin{cases}
                \ \;2m+n &=-1,\\
                \quad m &=2,\\
                \ \;2m+5n &=0,\\ 
                -4m+2n &=3.
            \end{cases} 
            \quad \Rightarrow
            \mbox{无解,向量组线性无关.}
        \]
        \item [(4)]$a_1=(1,-1,0,0),a_2=(0,1,-1,0),a_3=(0,0,1,-1),a_4=(-1,0,0,1)$.
        
        假设有$a_4=m\cdot a_1+n\cdot a_2+p\cdot a_3$,则
        \[
            \begin{cases}
                m &=-1,\\
                -m+n &=0,\\
                -n+p &=0,\\
                -p &=1.
            \end{cases}
            \quad \Rightarrow
            \begin{cases}
                m &=-1,\\
                n &=-1,\\
                p &=-1.
            \end{cases}
            \quad \Rightarrow
            a_4=-a_1-a_2-a_3,\mbox{向量组线性相关.}
        \]
    \end{enumerate}
    \item [12.]下列说法是否正确?为什么?
    \begin{enumerate}
        \item [(1)]错误.例如取$\alpha_1=(1,1),\alpha_2=(0,0)$,有$\alpha_2=0\cdot \alpha_1$,线性相关.
        但不存在$\lambda$,使得$\alpha_1=\lambda\cdot \alpha_2$.
        \item [(2)]错误.例如取$\alpha_1=(1,0),\alpha_2=(0,1),\alpha_3=(1,1)$,
        有$\left\{\alpha_1,\alpha_2\right\},\left\{\alpha_1,\alpha_3\right\}\left\{\alpha_2,\alpha_3\right\}$线性无关,
        但有$\alpha_3=\alpha_1+\alpha_2$.
        \item [(5)]错误.例如取$s=2$,则$\alpha_1+\alpha_2,\alpha_2+\alpha_1$线性相关.
        \item [(6)]正确.
        \item []若$s$为偶数,
        则$(\alpha_1+\alpha_2)+(\alpha_3+\alpha_4)+\cdots+(\alpha_{s-1}+\alpha_s)=(\alpha_2+\alpha_3)+(\alpha_4+\alpha_5)+\cdots+(\alpha_s+\alpha_1)$,线性相关;
        \item []若$s$为奇数,取$\alpha_{i+s}=\alpha_i\ (1\leqslant i\leqslant s)$.则
        $\alpha_i=\sum\limits_{k=0}^{s-1} {(-1)}^{k} (\alpha_{i+k}+\alpha_{i+k+1})$,两向量组等价.
        
        即$\mbox{rank}\left\{\alpha_1+\alpha_2,\ldots,\alpha_s+\alpha_1\right\}=\mbox{rank}\left\{\alpha_1,\ldots,\alpha_s \right\}<s$,即证线性相关.
    \end{enumerate}
    \item [15.]证明:非零向量组$\alpha_1,\ldots,\alpha_s$线性无关的充要条件是,每个$\alpha_i(1<i\leqslant s)$都不能用它面前的向量线性表示.
    \begin{proof}
        \begin{enumerate}
            \item []
            \item [(1)]必要性:显然.
            \item [(2)]充分性:假设有非零向量组$\alpha_1,\ldots,\alpha_s$每个$\alpha_i(1<i\leqslant s)$都不能用它面前的向量线性表示,且线性相关.
            
            由$\alpha_1,\ldots,\alpha_s$线性相关,得$\exists\ \alpha_i=\sum\limits_{j\neq i} \lambda_j \alpha_j$,且$\lambda_j$不全为0.

            取$k=\max \left\{\ j\ | \  \lambda_j\neq 0\right\}$,
            \begin{enumerate}
                \item [(a)]$k<i$.矛盾,不存在这样的非零向量组.
                \item [(b)]$k>i$.则有$\alpha_k=\alpha_i-\sum\limits_{j\neq k,i} \alpha_j$,矛盾,不存在这样的非零向量组.
            \end{enumerate}
            即假设不成立,充分性即证.
        \end{enumerate}
        综上,即证非零向量组$\alpha_1,\ldots,\alpha_s$线性无关的充要条件是,每个$\alpha_i(1<i\leqslant s)$都不能用它面前的向量线性表示.
    \end{proof}
    \item [16.]设向量组$\alpha_1,\ldots,\alpha_s$线性无关,$\beta=\lambda_1 \alpha_1 +\cdots+\lambda_s \alpha_s$.
    如果$\lambda_i \neq 0$,则用$\beta$代替$\alpha_i$后,向量组$\alpha_1,\ldots,\alpha_{i-1},\beta,\alpha_{i+1},\ldots,\alpha_s$线性无关.
    \begin{proof}
        \begin{align*}
            &\alpha_j=\begin{cases}
                & \alpha_i,\\
                & \\
                & \displaystyle{\frac{1}{\lambda_i}}\left(\beta-\left(\lambda_1 \alpha_1+\cdots+\lambda_{i-1} \alpha_{i-1}+\lambda_{i+1} \alpha_{i+1}+\cdots+\lambda_s \alpha_s\right)\right).            
            \end{cases}\\
            & \\
            \Rightarrow
            &\left\{\alpha_1,\ldots,\alpha_s\right\}\mbox{与}
            \left\{\alpha_1,\ldots,\alpha_{i-1},\beta,\alpha_{i+1},\ldots,\alpha_s\right\}
            \mbox{等价};\\
            & \\
            \Rightarrow
            &\mbox{rank}\left\{\alpha_1,\ldots,\alpha_s\right\}=
            \mbox{rank}\left\{\alpha_1,\ldots,\alpha_{i-1},\beta,\alpha_{i+1},\ldots,\alpha_s\right\}=s;\\
            & \\
            \Rightarrow
            &\alpha_1,\ldots,\alpha_{i-1},\beta,\alpha_{i+1},\ldots,\alpha_s\mbox{线性无关}.
        \end{align*}
    \end{proof}
\end{enumerate}
\end{document} 