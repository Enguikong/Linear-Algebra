\documentclass{article}
\usepackage{xeCJK,amsmath,geometry,graphicx,amssymb,zhnumber,booktabs,setspace,tasks,verbatim,amsthm,amsfonts,mathdots}
\geometry{a4paper,scale=0.8}   
\title{线性代数homework (第三周)}
\author{PB20000113孔浩宇}
\date{\today}
\begin{document}
\maketitle
\section{周二}
\subsection{习题五}
\begin{enumerate}
    \item [17.]设向量组$\alpha_1,\ldots,\alpha_r$线性无关,且$\alpha_1,\ldots,\alpha_r$可以由向量组$\beta_1,\ldots,\beta_r$线性表示,
    则$\beta_1,\ldots,\beta_r$也线性无关.
    \begin{proof}
        \[
            r=
            \mbox{rank}(\alpha_1,\ldots,\alpha_r)
            \leqslant
            \mbox{rank}(\beta_1,\ldots,\beta_r)
            \leqslant r 
            \Rightarrow \mbox{rank}(\beta_1,\ldots,\beta_r)=r.
            \]
    \end{proof}
    \item [19.]求下列向量组的极大无关组与秩:\begin{enumerate}
        \item [(3)]
        $\overrightarrow{a_1}=(0,1,2,3),\quad \overrightarrow{a_2}=(1,2,3,4),\quad \overrightarrow{a_3}=(3,4,5,6),\quad \overrightarrow{a_4}=(4,3,2,1),\quad \overrightarrow{a_5}=(6,5,4,3)$.
    \end{enumerate}
    \item [解]:记极大无关组为$S$.
    \begin{enumerate}
            \item [(a)]$S_1=\left\{\alpha_1\right\}$.
            \item [(b)]假设存在$x$使得$x\cdot \alpha_1=\alpha_2$,则$0\cdot x=1$,显然无解,故取$S_2=S_1\cup \left\{\alpha_2\right\}=\left\{\alpha_1,\alpha_2\right\}$.
            \item [(c)]假设存在$x,y$使得$x\cdot \alpha_1+y\cdot \alpha_2=\alpha_3$,则有
            \[
                \begin{cases}
                    \ \qquad \quad y &=3,\\
                    \ \ \;x+2y &=4,\\
                    \ 2x+3y &=5,\\
                    \ 3x+4y &=6.
                \end{cases}
                \Rightarrow
                \begin{cases}
                    \ x&=-2\\
                    \ y&=3.
                \end{cases}
                \quad
                \Rightarrow
                S_3=S_2=\left\{\alpha_1,\alpha_2\right\}.
            \]
            \item [(d)]假设存在$x,y$使得$x\cdot \alpha_1+y\cdot \alpha_2=\alpha_4$,则有
            \[
                \begin{cases}
                    \ \qquad \quad y &=4,\\
                    \ \ \;x+2y &=3,\\
                    \ 2x+3y &=2,\\
                    \ 3x+4y &=1.
                \end{cases}
                \Rightarrow
                \begin{cases}
                    \ x&=-5\\
                    \ y&=4.
                \end{cases}
                \quad
                \Rightarrow
                S_4=S_3=\left\{\alpha_1,\alpha_2\right\}.
            \]
            \item [(e)]假设存在$x,y$使得$x\cdot \alpha_1+y\cdot \alpha_2=\alpha_5$,则有
            \[
                \begin{cases}
                    \ \qquad \quad y &=6,\\
                    \ \ \;x+2y &=5,\\
                    \ 2x+3y &=4,\\
                    \ 3x+4y &=3.
                \end{cases}
                \Rightarrow
                \begin{cases}
                    \ x&=-7\\
                    \ y&=6.
                \end{cases}
                \quad
                \Rightarrow
                S_5=S_4=\left\{\alpha_1,\alpha_2\right\}.
            \]
            \item []综上,极大无关组$S=\left\{\alpha_1,\alpha_2\right\},\mbox{rank}(\alpha_1,\ldots,\alpha_6)=2$.
        \end{enumerate}
    \item [22.]设向量组$\alpha_1,\ldots,\alpha_m$的秩为$r$,则其中任何$r$个线性无关的向量构成$\alpha_1,\ldots,\alpha_m$的极大无关组.
    \begin{proof}
        设有任何$r$个线性无关的向量$\alpha_{i1},\ldots,\alpha_{ir}$构成一个向量组$S_1$,$\forall\ \alpha_{j}\notin S_1,S_2=S_1\cup \left\{\alpha_{j}\right\}$.
        \[
            r+1>r=\mbox{rank}(\alpha_1,\ldots,\alpha_m)
            \Rightarrow
            S_2\mbox{线性相关},\mbox{又}S_1\mbox{线性无关}
            \Rightarrow
            S_1\mbox{构成}\alpha_1,\ldots,\alpha_m\mbox{的极大无关组}.
        \]
    \end{proof}
    \item [23.]设向量组$\alpha_1,\ldots,\alpha_m$的秩为$r$,如果$\alpha_1,\ldots,\alpha_m$可以由它的$r$个向量线性表示,则这$r$个向量构成$\alpha_1,\ldots,\alpha_m$的极大无关组.
    \begin{proof}将这$r$个向量记为$\alpha_{i1},\ldots,\alpha_{ir}$.
        \begin{align*}
            & r=\mbox{rank}(\alpha_1,\ldots,\alpha_m)
            \leqslant
            \mbox{rank}(\alpha_{i1},\ldots,\alpha_{ir})
            \leqslant r
            \Rightarrow
            \mbox{rank}(\alpha_{i1},\ldots,\alpha_{ir})=r\\
            \xrightarrow{\mbox{T22结论}}
            & \mbox{rank}(\alpha_{i1},\ldots,\alpha_{ir})\mbox{构成}
            \alpha_1,\ldots,\alpha_m\mbox{的极大无关组}.
        \end{align*}
    \end{proof}
\end{enumerate}
\section{周四}
\subsection{习题五}
\begin{enumerate}
    \item [20.]求下列矩阵的秩,并求出它的行向量空间的一组基.
    
    (2) 进行初等变换:
    \[
        \begin{pmatrix}
            3  &  6 &  1 &  5\\
            1  &  4 & -1 &  3\\
            -1 & -10&  5 & -7\\
            4  & -2 &  8 &  0
        \end{pmatrix}
        \to
        \begin{pmatrix}
            0 & -6 &  4 & -6\\
            1 &  4 & -1 &  3\\
            0 & -6 &  4 & -4\\
            0 & -18& 12 & -12
        \end{pmatrix}
        \to
        \begin{pmatrix}
            0 &  0 &  0 & -2\\
            1 &  0 &  0 &  0\\
            0 & -6 &  4 & -4\\
            0 &  0 &  0 & 0
        \end{pmatrix}
        \to
        \begin{pmatrix}
            1 & 0 & 0 & 0\\
            0 & 4 & 0 & 0\\
            0 & 0 & -2& 0\\
            0 & 0 & 0 & 0
        \end{pmatrix}
    \]
    即
    \[
        \mbox{rank}
        \begin{pmatrix}
            3  &  6 &  1 &  5\\
            1  &  4 & -1 &  3\\
            -1 & -10&  5 & -7\\
            4  & -2 &  8 &  0
        \end{pmatrix}
        =\mbox{rank}
        \begin{pmatrix}
            1 & 0 & 0 & 0\\
            0 & 4 & 0 & 0\\
            0 & 0 & -2& 0\\
            0 & 0 & 0 & 0
        \end{pmatrix}
        =3.
    \]
    记原矩阵为${(\alpha_1\ \alpha_2\ \alpha_3\ \alpha_4)}^T$,由初等变换可得$\alpha_4 -4\cdot \alpha_2=3\cdot(\alpha_2+\alpha_3)$
    \item [24.]证明:$\mbox{rank}(\alpha_1,\ldots,\alpha_r,\beta_1,\ldots,\beta_s)\leqslant\mbox{rank}(\alpha_1,\ldots,\alpha_r)+\mbox{rank}(\beta_1,\ldots,\beta_s)$.
    \begin{proof}记$A=(\alpha_1,\ldots,\alpha_r),B=(\beta_1,\ldots,\beta_s)$.
        \[
            \mbox{rank}(\alpha_1,\ldots,\alpha_r,\beta_1,\ldots,\beta_s)
            =\mbox{rank}\begin{pmatrix}A & B \end{pmatrix}
            \leqslant
            \mbox{rank}(A)+\mbox{rank}(B)
            =\mbox{rank}(\alpha_1,\ldots,\alpha_r)+\mbox{rank}(\beta_1,\ldots,\beta_s)
        \]
    \end{proof}
    \item [27.]设$A,B$是同阶矩阵.证明:$\mbox{rank}(A+B)\leqslant\mbox{rank}(A)+\mbox{rank}(B)$.
    \begin{proof}
        \[
            \mbox{rank}(A+B)\leqslant
            \max\left\{\mbox{rank}(A),\mbox{rank}(B),\mbox{rank}(A+B)\right\}
            \leqslant
            \mbox{rank}\begin{pmatrix}
                A & B
            \end{pmatrix}
            \leqslant
            \mbox{rank}(A)+\mbox{rank}(B)
        \]
    \end{proof}
    \item [34.]以向量组$\alpha_1=(3,1,0),\alpha_2=(6,3,2),\alpha_3=(1,3,5)$为基,求$\beta=(2,-1,2)$的坐标.
    
    设$\beta=x\cdot \alpha_1+y\cdot \alpha_2+z\cdot \alpha_3$,有
    \[
        \begin{cases}
            \ 3x+6y+\ \;z & =2\\
            \ \ \;x+3y+3z & =-1\\
            \ \qquad\ 2y+5z & =2
        \end{cases}
        \quad \Rightarrow
        \begin{cases}
            \ x & =-76\\
            \ y & =\ 41\\
            \ z & =-16
        \end{cases}
        ,\mbox{即}\beta=-76\alpha_1+41\alpha_2-16\alpha_3.
    \]
    \item [35.]设$\alpha_1=(3,2,-1,4),\alpha_2=(2,3,0,-1)$.
    \begin{enumerate}
        \item [(1)]将$\alpha_1,\alpha_2$扩充为$\mathbb{R}^4$的一组基;
        \item [(2)]给出标准基在该组基下的表示;
        \item [(3)]求$\beta=(1,3,4,-2)$在该组基下的坐标.
    \end{enumerate}
    解:
    \begin{enumerate}
        \item [(1)]显然$\alpha_1,\alpha_2$线性无关,取$\alpha_3=(0,0,1,0),\alpha_4=(0,0,0,1)$,有
        \[
            \mbox{rank}(\alpha_1,\alpha_2,\alpha_3,\alpha_4)=
            \mbox{rank}\begin{pmatrix}
                3 & 2 & -1 & 4\\
                2 & 3 &  0 & -1\\
                0 & 0 &  1 & 0\\
                0 & 0 &  0 & 1
            \end{pmatrix}
            =\mbox{rank}
            \begin{pmatrix}
                3 & 0 & 0 & 0\\
                0 & 3 & 0 & 0\\
                0 & 0 & 1 & 0\\
                0 & 0 & 0 & 1
            \end{pmatrix}
            =3
        \]
        即$\alpha_1,\alpha_2,\alpha_3,\alpha_4$即为$\mathbb{R}^4$的一组基.
        \item [(2)]
        \[
            \begin{cases}
                \ e_1 & =\displaystyle{\frac{3}{5}\alpha_1-\frac{2}{5}\alpha_2+\frac{3}{5}\alpha_3-\frac{14}{5}\alpha_4}\\
                 & \\
                \ e_2 & =\displaystyle{-\frac{2}{5}\alpha_1+\frac{3}{5}\alpha_2-\frac{2}{5}\alpha_3+\frac{11}{5}\alpha_4}\\
                 & \\
                \ e_3 & =\alpha_3\\
                 & \\
                \ e_4 & =\alpha_4
            \end{cases}
        \]
        \item [(3)]
        \[
            \beta=(1,3,4,-2)
            \begin{pmatrix}
                e_1\\
                e_2\\
                e_3\\
                e_4
            \end{pmatrix}
            =(1,3,4,-2)
            \begin{pmatrix}
                \ \displaystyle{\frac{3}{5}} &\displaystyle{-\frac{2}{5}}  & \ \displaystyle{\frac{3}{5}} & \displaystyle{-\frac{14}{5}}\\
                & & & \\
                \displaystyle{-\frac{2}{5}} & \ \displaystyle{\frac{3}{5}} & \displaystyle{-\frac{2}{5}} & \displaystyle{\frac{11}{5}}\\
                & & & \\
                0 & 0 & 1 & 0\\
                & & & \\
                0 & 0 & 0 & 1
            \end{pmatrix}
            \begin{pmatrix}
                \alpha_1\\
                \alpha_2\\
                \alpha_3\\
                \alpha_4
            \end{pmatrix}
            =\left(-\frac{3}{5},\frac{7}{5},\frac{17}{5},\frac{9}{5}\right)
            \begin{pmatrix}
                \alpha_1\\
                \alpha_2\\
                \alpha_3\\
                \alpha_4
            \end{pmatrix}.
        \]
        即$\beta=\displaystyle{-\frac{3}{5}\alpha_1+\frac{7}{5}\alpha_2+\frac{17}{5}\alpha_3+\frac{9}{5}\alpha_4}$.
    \end{enumerate}
\end{enumerate}
\end{document}