\documentclass{article}
\usepackage{xeCJK,amsmath,geometry,graphicx,amssymb,zhnumber,booktabs,setspace,tasks,verbatim,amsthm,amsfonts,mathdots}
\geometry{a4paper,scale=0.8}   
\title{线性代数homework (第三周)}
\author{PB20000113孔浩宇}
\date{\today}
\begin{document}
\maketitle
\section{周二}
\subsection{习题五}
\begin{enumerate}
    \item [17.]设向量组$\alpha_1,\ldots,\alpha_r$线性无关,且$\alpha_1,\ldots,\alpha_r$可以由向量组$\beta_1,\ldots,\beta_r$线性表示,
    则$\beta_1,\ldots,\beta_r$也线性无关.
    \begin{proof}
        \[
            r=
            \mbox{rank}(\alpha_1,\ldots,\alpha_r)
            \leqslant
            \mbox{rank}(\beta_1,\ldots,\beta_r)
            \leqslant r 
            \Rightarrow \mbox{rank}(\beta_1,\ldots,\beta_r)=r.
            \]
    \end{proof}
    \item [19.]求下列向量组的极大无关组与秩:\begin{enumerate}
        \item [(3)]
        $\overrightarrow{a_1}=(0,1,2,3),\quad \overrightarrow{a_2}=(1,2,3,4),\quad \overrightarrow{a_3}=(3,4,5,6),\quad \overrightarrow{a_4}=(4,3,2,1),\quad \overrightarrow{a_5}=(6,5,4,3)$.
    \end{enumerate}
    \item [解]:记极大无关组为$S$.
    \begin{enumerate}
            \item [(a)]$S_1=\left\{\alpha_1\right\}$.
            \item [(b)]假设存在$x$使得$x\cdot \alpha_1=\alpha_2$,则$0\cdot x=1$,显然无解,故取$S_2=S_1\cup \left\{\alpha_2\right\}=\left\{\alpha_1,\alpha_2\right\}$.
            \item [(c)]假设存在$x,y$使得$x\cdot \alpha_1+y\cdot \alpha_2=\alpha_3$,则有
            \[
                \begin{cases}
                    \ \qquad \quad y &=3,\\
                    \ \ \;x+2y &=4,\\
                    \ 2x+3y &=5,\\
                    \ 3x+4y &=6.
                \end{cases}
                \Rightarrow
                \begin{cases}
                    \ x&=-2\\
                    \ y&=3.
                \end{cases}
                \quad
                \Rightarrow
                S_3=S_2=\left\{\alpha_1,\alpha_2\right\}.
            \]
            \item [(d)]假设存在$x,y$使得$x\cdot \alpha_1+y\cdot \alpha_2=\alpha_4$,则有
            \[
                \begin{cases}
                    \ \qquad \quad y &=4,\\
                    \ \ \;x+2y &=3,\\
                    \ 2x+3y &=2,\\
                    \ 3x+4y &=1.
                \end{cases}
                \Rightarrow
                \begin{cases}
                    \ x&=-5\\
                    \ y&=4.
                \end{cases}
                \quad
                \Rightarrow
                S_4=S_3=\left\{\alpha_1,\alpha_2\right\}.
            \]
            \item [(e)]假设存在$x,y$使得$x\cdot \alpha_1+y\cdot \alpha_2=\alpha_5$,则有
            \[
                \begin{cases}
                    \ \qquad \quad y &=6,\\
                    \ \ \;x+2y &=5,\\
                    \ 2x+3y &=4,\\
                    \ 3x+4y &=3.
                \end{cases}
                \Rightarrow
                \begin{cases}
                    \ x&=-7\\
                    \ y&=6.
                \end{cases}
                \quad
                \Rightarrow
                S_5=S_4=\left\{\alpha_1,\alpha_2\right\}.
            \]
            \item []综上,极大无关组$S=\left\{\alpha_1,\alpha_2\right\},\mbox{rank}(\alpha_1,\ldots,\alpha_6)=2$.
        \end{enumerate}
    \item [22.]设向量组$\alpha_1,\ldots,\alpha_m$的秩为$r$,则其中任何$r$个线性无关的向量构成$\alpha_1,\ldots,\alpha_m$的极大无关组.
    \begin{proof}
        设有任何$r$个线性无关的向量$\alpha_{i1},\ldots,\alpha_{ir}$构成一个向量组$S_1$,$\forall\ \alpha_{j}\notin S_1,S_2=S_1\cup \left\{\alpha_{j}\right\}$.
        \[
            r+1>r=\mbox{rank}(\alpha_1,\ldots,\alpha_m)
            \Rightarrow
            S_2\mbox{线性相关},\mbox{又}S_1\mbox{线性无关}
            \Rightarrow
            S_1\mbox{构成}\alpha_1,\ldots,\alpha_m\mbox{的极大无关组}.
        \]
    \end{proof}
    \item [23.]设向量组$\alpha_1,\ldots,\alpha_m$的秩为$r$,如果$\alpha_1,\ldots,\alpha_m$可以由它的$r$个向量线性表示,则这$r$个向量构成$\alpha_1,\ldots,\alpha_m$的极大无关组.
    \begin{proof}将这$r$个向量记为$\alpha_{i1},\ldots,\alpha_{ir}$.
        \begin{align*}
            & r=\mbox{rank}(\alpha_1,\ldots,\alpha_m)
            \leqslant
            \mbox{rank}(\alpha_{i1},\ldots,\alpha_{ir})
            \leqslant r
            \Rightarrow
            \mbox{rank}(\alpha_{i1},\ldots,\alpha_{ir})=r\\
            \xrightarrow{\mbox{T22结论}}
            & \mbox{rank}(\alpha_{i1},\ldots,\alpha_{ir})\mbox{构成}
            \alpha_1,\ldots,\alpha_m\mbox{的极大无关组}.
        \end{align*}
    \end{proof}
\end{enumerate}
\section{周四}
\subsection{习题五}
\end{document}