\documentclass{article}
\usepackage{xeCJK,amsmath,geometry,graphicx,amssymb,zhnumber,booktabs,setspace,tasks,verbatim,amsthm,amsfonts,mathdots}
\geometry{a4paper,scale=0.8}   
\title{线性代数homework (第九周)}
\author{PB20000113孔浩宇}
\date{\today}
\begin{document}
\maketitle
\section{周二}
\subsection{习题五}
\begin{enumerate}
    \item [37.]设$n$元$n-1$个方程的齐次线性方程组的系数阵$A$的秩为$n-1$,求该齐次线性方程组的基础解系.
    
    解:
    \begin{enumerate}
        \item []设$A={(a_{ij})}_{(n-1)*n}$,则基础解系个数为1;取方阵$B=\begin{pmatrix}A\\ \beta\end{pmatrix},\beta=(b_1,\ldots,b_n)$;
        \item []记$B$第$n$行对应的代数余子式为$B_1,\ldots,B_n$,则有$\sum\limits_{i=1}^n b_i\cdot B_i=\det(B)$;
        \item []取$\beta=(a_{i1},\ldots,a_{in})$,则$\mbox{rank}(A)=n-1,\sum\limits_{i=1}^n b_i\cdot B_i=\det(B)=0$;
        \item []即$\alpha={(B_1,\ldots,B_n)}^T$为该齐次线性方程组的基础解系.
    \end{enumerate}
    \item [38.]设$\alpha_1,\ldots,\alpha_s$为非齐次方程组$Ax=b$的一组解,$\lambda_1,\ldots,\lambda_s$为常数.
    给出$\lambda_1 \alpha_s+\cdots+\lambda_s \alpha_s$为该线性方程组的解的充要条件.
    \begin{align*}
        A(\lambda_1 \alpha_s+\cdots+\lambda_s \alpha_s)=b
        \Leftrightarrow
        &\quad \lambda_1\cdot A\alpha_1+\cdots+\lambda_s\cdot A\alpha_s=b;\\
        \Leftrightarrow
        &\quad (\lambda_1+\cdots+\lambda_s)b=b;\\
        \Leftrightarrow
        &\quad \lambda_1+\cdots+\lambda_s=1.
    \end{align*}
    \item [40.]求下列齐次线性方程组的基础解系与通解:
    \[
        (2)
        \begin{cases}
            \quad x_1 + \,\;x_2 +\,\; x_3 +\;\; x_4 - 4x_5 &=0\\
            \quad x_1 - 2x_2 + 3x_3 -4x_4 +2x_5 &=0\\
            \;-x_1 + 3x_2 - 5x_3 + 7x_4 - 4x_5 &=0\\
            \quad x_1 + 2x_2 -\,\; x_3 +\, 4x_4 - 6x_5 &=0
        \end{cases}
        \Leftrightarrow
        \begin{pmatrix}
            1 & 1 & 1 & 1 & -4\\
            1 &-2 & 3 &-4 & 2\\
            -1& 3 &-5 & 7 & -4\\
            1 & 2 &-1 & 4 & -6
        \end{pmatrix}
        \begin{pmatrix}
            x_1\\
            x_2\\
            x_3\\
            x_4\\
            x_5
        \end{pmatrix}
        =0.
    \]
    对增广矩阵进行线性变换:
    \begin{align*}
        &\begin{pmatrix}
            1 & 1 & 1 & 1 & -4\\
            1 &-2 & 3 &-4 & 2\\
            -1& 3 &-5 & 7 & -4\\
            1 & 2 &-1 & 4 & -6
        \end{pmatrix}
        \to
        \begin{pmatrix}
            1 & 1 & 1 & 1 & -4\\
            0 &-3 & 2 &-5 & 6\\
            0 & 4 &-4 & 8 & -8\\
            0 & 1 &-2 & 3 & -2
        \end{pmatrix}
        \to
        \begin{pmatrix}
            1 & 1 & 1 & 1 & -4\\
            0 &-3 & 2 &-5 & 6\\
            0 & 1 &-1 & 2 & -2\\
            0 & 1 &-2 & 3 & -2
        \end{pmatrix}\\
        &\\
        \to &
        \begin{pmatrix}
            1 & 1 & 1 & 1 & -4\\
            0 & 0 &-1 & 1 & 0\\
            0 & 1 &-1 & 2 & -2\\
            0 & 0 &-1 & 1 & 0
        \end{pmatrix}
        \to
        \begin{pmatrix}
            1 & 1 & 1 & 1 & -4\\
            0 & 1 &-1 & 2 & -2\\
            0 & 0 &-1 & 1 & 0\\
            0 & 0 & 0 & 0 & 0
        \end{pmatrix}
        \to
        \begin{pmatrix}
            1 & 0 & 0 & 1 & -2\\
            0 & 1 & 0 & 1 & -2\\
            0 & 0 &-1 & 1 & 0\\
            0 & 0 & 0 & 0 & 0
        \end{pmatrix}
    \end{align*}
    即原方程组等价于
    \[
        \begin{pmatrix}
            1 & 0 & 0 & 1 & -2\\
            0 & 1 & 0 & 1 & -2\\
            0 & 0 &-1 & 1 & 0\\
            0 & 0 & 0 & 0 & 0
        \end{pmatrix}
        \begin{pmatrix}
            x_1\\
            x_2\\
            x_3\\
            x_4\\
            x_5
        \end{pmatrix}=0
        \Leftrightarrow
        \begin{cases}
            x_1 + x_4 - 2x_5 &=0\\
            x_2 + x_4 - 2x_5 &=0\\
            \quad \ \,-x_3 + x_4 &=0
        \end{cases}
        \xrightarrow[x_5=b]{x_4=a}
        \begin{pmatrix}
            x_1\\
            x_2\\
            x_3\\
            x_4\\
            x_5
        \end{pmatrix}=
        \begin{pmatrix}
            -a+2b\\
            -a+2b\\
            a\\
            a\\
            b
        \end{pmatrix}.
    \]
    即基础解系为${(-1,-1,1,1,0)}^T,{(2,2,0,0,1)}^T$;通解为${(-a+2b,-a+2b,a,a,b)}^T$.
    \item [41.]已知$\mathbb{F}^5$中向量$\eta _1={(1,2,3,2,1)}^T$及$\eta _2={(1,3,2,1,2)}^T$.找一个齐次线性方程组,使得$\eta_1$与$\eta_2$为该方程组的基础解系.
    \[
        A
        \begin{pmatrix}
            \eta_1 & \eta_2
        \end{pmatrix}=0
        \Leftrightarrow
        \begin{pmatrix}
            \eta_1 ^T\\
            \eta_2 ^T
        \end{pmatrix}
        A^T=0
        \Leftrightarrow
        \begin{pmatrix}
            1 & 2 & 3 & 2 & 1\\
            1 & 3 & 2 & 1 & 2
        \end{pmatrix}
        \begin{pmatrix}
            a_{11} & a_{21} & a_{31}\\
            a_{12} & a_{22} & a_{32}\\
            a_{13} & a_{23} & a_{33}\\
            a_{14} & a_{24} & a_{34}\\
            a_{15} & a_{25} & a_{35}\\
        \end{pmatrix}
        =O.
    \]
    \[
        \begin{pmatrix}
            1 & 2 & 3 & 2 & 1\\
            1 & 3 & 2 & 1 & 2
        \end{pmatrix}
        \begin{pmatrix}
            a_1\\ a_2\\ a_3\\ a_4\\ a_5
        \end{pmatrix}=O
        \Leftrightarrow
        \begin{pmatrix}
            1 & 0 & 5 & 4 & 1\\
            0 & 1 &-1 &-1 & 1
        \end{pmatrix}
        \begin{pmatrix}
            a_1\\ a_2\\ a_3\\ a_4\\ a_5
        \end{pmatrix}=O
        \Leftrightarrow
        \begin{pmatrix}
            a_1\\ a_2\\ a_3\\ a_4\\ a_5
        \end{pmatrix}=
        \begin{pmatrix}
            -5a_3-4a_4-a_5 \\ a_3+a_4-a_5 \\ a_3 \\ a_4 \\ a_5
        \end{pmatrix}
    \]
    即$\begin{pmatrix}\eta_1 ^T\\\eta_2 ^T\end{pmatrix} \alpha=0$基础解系为${(-5,1,1,0,0)}^T,{(-4,1,0,1,0)}^T,{(-1,-1,0,0,1)}^T$.可得
    \[
        A=\begin{pmatrix}
            -5 & 1 & 1 & 0 & 0\\
            -4 & 1 & 0 & 1 & 0\\
            1 & -1 & 0 & 0 & 1
        \end{pmatrix}.
    \]
    \item [43.]判断下列集合关于规定的运算是否构成线性空间:
    \begin{enumerate}
        \item [(1)]$V$是所有实数对$(x,y)$的集合,数域$F=\mathbb{R}$.定义\[(x_1,y_1)+ (x_2,y_2)= (x_1+x_2,y_1 + y_2), \lambda(x,y)= (x,y).\]
        \item [(2)]$V$是所有满足$f(-1)=0$的实函数的集合,数域$F=\mathbb{R}$.定义加法为函数的加法,数乘为数与函数的乘法.
        \item [(3)]$V$是所有满足$f(0)\neq 0$的实函数的集合,数域$F=\mathbb{R}$.定义加法为函数的加法,数乘为数与函数的乘法.
        \item [(4)]$V$是数域$F$上所有$n$阶可逆方阵的全体,加法为矩阵的加法,数乘为矩阵的数乘.
    \end{enumerate}
    解:
    \begin{enumerate}
        \item [(1)]不构成线性空间.$\lambda (x,y)+\mu (x,y)=(2x,2y)\neq (\lambda+\mu)(x,y)\quad(x,y\neq 0)$
        \item [(2)]构成线性空间
        \begin{align*}
            \forall\ f,g,h\in V;\lambda,\mu \in \mathbb{R}:
            &\ (A1)\ f+g=g+f,\mbox{且}(f+g)(-1)=f(-1)+g(-1)=0\Rightarrow f+g\in V;\\
            &\ (A2)\ (f+g)+h=f+(g+h);\\
            &\ (A3)\ \exists\ O=0,f+O=O+f=f;\\
            &\ (A4)\ \exists\ p,\forall\ x\in\mathbb{R},p(x)=-f(x),f+p=p+f=O=0;\\
            &\ (D1)\ \lambda(f+g)=\lambda f+\lambda g,\mbox{且} (\lambda f)(-1)=\lambda f(-1)=0\Rightarrow \lambda f\in V;\\
            &\ (D2)\ (\lambda+\mu)f=\lambda f+\mu f;\\
            &\ (D3)\ \lambda(\mu f)=(\lambda \mu)f;\\
            &\ (D4)\ 1f=f.
        \end{align*}
        \item [(3)]不构成线性空间.$\forall\ f\in V,\mbox{显然}-f\in V,f+(-f)=0\Rightarrow f+(-f)\notin V$,对加法不封闭.
        \item [(4)]不构成线性空间.$\forall\ A\in V,\mbox{显然}-A\in V,A+(-A)=O\Rightarrow A+(-A)\notin V$,对加法不封闭.
    \end{enumerate}
\end{enumerate}
\section{周四}
\subsection{习题五}
\begin{enumerate}
    \item [44.]设V是所有实函数全体在实数域上构成的线性空间,判断下列函数组是否线性相关.
    \begin{enumerate}
        \item [(1)]$1,x, \sin x$;
        \item [(2)]$1,x,e^x$;
        \item [(3)]$1, \cos 2x, \cos ^2 x$;
        \item [(4)]$1,x^2,{(x-1)}^3,{(x+1)}^3$;
        \item [(5)]$\cos x,\cos 2x,\ldots,\cos nx$.
    \end{enumerate}
    解:
    \begin{enumerate}
        \item [(1)]线性无关.若存在$a,b,c\in \mathbb{R},a+bx+c\sin x\equiv 0$,取$x=0,x=\frac{\pi}{2},x=\pi$,则
        \[a=a+b\pi=a+\frac{b\pi}{2}+c=0\Rightarrow a=b=c=0.\]
        \item [(2)]线性无关.若存在$a,b,c\in R,a+bx+c\cdot e^x\equiv 0$,取$x=0,1,-1$,则
        \[a+c=a+b+c\cdot e=a-b+\frac{c}{e}=0\Rightarrow a=b=c=0.\]
        \item [(3)]线性相关.$\cos 2x=2\cos^2 x -1=0$.
        \item [(4)]线性相关.${(x+1)}^3={(x-1)}^3 +6x^2+2$.
        \item [(5)]线性无关.若存在$a_1,a_2,\ldots,a_n \in \mathbb{R},\sum\limits_{i=1}^n a_i\cdot \cos ix\equiv 0$,则记$f(x)=\sum\limits_{i=1}^n a_i\cdot \cos ix$.
        \[
            f(x)=0 \Leftrightarrow \forall\ k\geqslant 0,f^{(k)}(x)=0 \Rightarrow f(x)=f^{(4)}(x)=\cdots=f^{(4\cdot(n-1))}(x)=0
        \]
        即
        \[
            \begin{pmatrix}
                1 & 1 & \cdots & 1\\
                1 & 2^4 & \cdots & n^4\\
                1 & {(2^4)}^2 & \cdots & {(n^4)}^2\\
                \vdots & \vdots & & \vdots\\
                1 & {(2^4)}^{n-1} & \cdots & {(n^4)}^{n-1}
            \end{pmatrix}
            \begin{pmatrix}
                a_1\cdot \cos x\\
                a_2\cdot \cos 2x \\
                \vdots\\
                a_n\cdot \cos nx
            \end{pmatrix}=O
            \xrightarrow{\mbox{由$Vandermonde$矩阵}}
            \begin{pmatrix}
                a_1\cdot \cos x\\
                a_2\cdot \cos 2x \\
                \vdots\\
                a_n\cdot \cos nx
            \end{pmatrix}=O.
        \]
        显然有$a_1=a_2=\cdots=a_n=0$.
    \end{enumerate}
    \item [46.]设$F_n [x]$是次数小于或等于n的多项式全体构成的线性空间.
    \begin{enumerate}
        \item [(1)]证明:$ S=\left\{1,x-1,{(x-1)}^2,\ldots,{(x-1)}^n\right\}$构成$\mathbb{F}^n [x]$的一组基;
        \item [(2)]求$S$到基$T=\left\{1,x,\ldots,x^n\right\}$之间的过渡矩阵;
        \item [(3)]求多项式$p(x)=a_0+a_1 x+\cdots+a_n x^n \in \mathbb{F}[x]$在基$S$下的坐标.
    \end{enumerate}
    \begin{proof}
        \begin{enumerate}
            \item []
            \item [(1)]
            \[
                {\begin{pmatrix}
                    1 \\ x-1 \\ {(x-1)}^2 \\ \vdots \\{(x-1)}^n
                \end{pmatrix}}^T
                =
                {\begin{pmatrix}
                    1 \\ x \\ x^2 \\ \ldots \\ x^n
                \end{pmatrix}}^T
                \begin{pmatrix}
                    C_{0}^{0}\cdot{(-1)}^{0} & C_{1}^{0}\cdot {(-1)}^{1} & C_{2}^{0}\cdot {(-1)}^{2} & \cdots & C_{n}^{0}\cdot {(-1)}^{n}\\
                    0 & C_{1}^{1}\cdot {(-1)}^{0} & C_{2}^{1}\cdot {(-1)}^{1} &\cdots &  C_{n}^{1}\cdot {(-1)}^{n-1}\\
                    0 & 0 & C_{2}^{2}\cdot {(-1)}^{0} & \cdots & C_{n}^{2}\cdot {(-1)}^{n-2}\\
                    \vdots & \vdots & \vdots & & \vdots\\
                    0 & 0 & 0 & \cdots & C_{n}^{n}\cdot {(-1)}^{0}
                \end{pmatrix}.
            \]
            由过渡矩阵
            \[
                \det(T)=\det\begin{pmatrix}
                    C_{0}^{0}\cdot{(-1)}^{0} & C_{1}^{0}\cdot {(-1)}^{1} & C_{2}^{0}\cdot {(-1)}^{2} & \cdots & C_{n}^{0}\cdot {(-1)}^{n}\\
                    0 & C_{1}^{1}\cdot {(-1)}^{0} & C_{2}^{1}\cdot {(-1)}^{1} &\cdots &  C_{n}^{1}\cdot {(-1)}^{n-1}\\
                    0 & 0 & C_{2}^{2}\cdot {(-1)}^{0} & \cdots & C_{n}^{2}\cdot {(-1)}^{n-2}\\
                    \vdots & \vdots & \vdots & & \vdots\\
                    0 & 0 & 0 & \cdots & C_{n}^{n}\cdot {(-1)}^{0}
                \end{pmatrix}=1,\Rightarrow
                \mbox{即证}.
            \]
            \item [(2)]由 (1)可得过渡矩阵为
            \[
                \begin{pmatrix}
                    C_{0}^{0}\cdot{(-1)}^{0} & C_{1}^{0}\cdot {(-1)}^{1} & C_{2}^{0}\cdot {(-1)}^{2} & \cdots & C_{n}^{0}\cdot {(-1)}^{n}\\
                    0 & C_{1}^{1}\cdot {(-1)}^{0} & C_{2}^{1}\cdot {(-1)}^{1} &\cdots &  C_{n}^{1}\cdot {(-1)}^{n-1}\\
                    0 & 0 & C_{2}^{2}\cdot {(-1)}^{0} & \cdots & C_{n}^{2}\cdot {(-1)}^{n-2}\\
                    \vdots & \vdots & \vdots & & \vdots\\
                    0 & 0 & 0 & \cdots & C_{n}^{n}\cdot {(-1)}^{0}
                \end{pmatrix}.
            \]
            \item [(3)]记$p(x)$在基$S$下的坐标为$(b_0,b_1,\ldots,b_n)$,则
            \[
                p(x)=\sum\limits_{i=0}^n b_i\cdot {(x-1)}^i
                \Rightarrow
                b_i= \displaystyle{\frac{1}{i!}} p^{(i)}(1)
                \Rightarrow
                (b_0,b_1,\ldots,b_n)=
                \left(\sum\limits_{i=0}^{n} C_{i}^{0} a_i,\sum\limits_{i=1}^{n} C_{i}^{1} a_i,\ldots,\sum\limits_{i=n}^n C_{i}^{n} a_n \right)
            \]
        \end{enumerate}
    \end{proof}
    \item [47.]$V$是数域$\mathbb{F}$上所有$n$阶对称矩阵的全体,定义加法为矩阵的加法,数乘为矩阵的数乘.证明: $V$是线性空间,并求$V$的一组基及维数. 
    \begin{proof}
        \begin{enumerate}
            \item []
            \item [(1)]证明$V$是线性空间:
            \begin{align*}
                \forall\ A,B,C\in V;\lambda,\mu \in \mathbb{R}:
                &\ (A1)\ A+B=B+A,\mbox{且}{(A+B)}^T=A^T+B^T=A+B \Rightarrow A+B\in V;\\
                &\ (A2)\ (A+B)+C=A+(B+C);\\
                &\ (A3)\ \exists\ O\in V,A+O=O+A=A;\\
                &\ (A4)\ \exists\ A'=-A,{(-A)}^T=-A\Rightarrow A'\in V,\mbox{且}A+A'=0;\\
                &\ (D1)\ \lambda(A+B)=\lambda A+\lambda B,\mbox{且} {(\lambda A)}^T=\lambda A^T=\lambda A \Rightarrow \lambda A\in V;\\
                &\ (D2)\ (\lambda+\mu)A=\lambda A+\mu A;\\
                &\ (D3)\ \lambda(\mu A)=(\lambda \mu)A;\\
                &\ (D4)\ 1A=A.
            \end{align*}
            \item [(2)]
            \[
                \forall\ A=(a_{ij})\in V,
                A=\sum\limits_{i=1}^n a_{ii}E_{ii}+\sum\limits_{i=1}^n \sum\limits_{j=i+1}^n a_{ij}(E_{ij}+E_{ji}).
            \]
            又$S=\left\{E_{ii}(1\leq i\leq n),E_{ij}+E_{ji}(1\leq i\leq j\leq n)\right\}$为线性无关组,故可取S为V的一组基.
            
            即
            \[
                \mbox{基}S=\left\{E_{ii}(1\leq i\leq n),E_{ij}+E_{ji}(1\leq i\leq j\leq n)\right\},\
                \mbox{rank}V=\mbox{rank}S=\displaystyle{\frac{n(n+1)}{2}}.
            \]
        \end{enumerate}
    \end{proof}
\end{enumerate}
\end{document}