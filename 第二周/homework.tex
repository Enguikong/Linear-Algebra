\documentclass{article}
\usepackage{xeCJK,amsmath,geometry,graphicx,extarrows}
\geometry{a4paper,scale=0.8}   
\title{线性代数homework (第二周)}
\author{PB20000113孔浩宇}
\date{\today}
\begin{document}
\maketitle
\section{周二}                             %——一号子标题  
    \subsection{习题三}                      %——二号子标题 
    \begin{enumerate}
        \item [1.]解下列线性方程组:
        \begin{enumerate}
            \item [(2)]
            \[
                \begin{cases}
                    \ x_1-2x_2+3x_3-4x_4&=4,\\
                    \ \qquad \  \ \, x_2-\ x_3+\ \, x_4&=-3,\\
                    \ x_1+\, 3x_2 \qquad \ \  -3x_4&=1,\\
                    \ \qquad -7x_2+3x_3+x_4&=-3.
                \end{cases}
            \]
            解:
            \[
            \left(\begin{array}{ccccc}
                1 & -2 & 3 & -4 & 4\\
                0 & 1 & -1 & 1 & -3\\
                1 & 3 & 0 & -3 & 1\\
                0 & -7 & 3 & 1 & -3 
                \end{array}
            \right)
            \xrightarrow{-r_1 \rightarrow r_3}
            \left(\begin{array}{ccccc}
                1 & -2 & 3 & -4 & 4\\
                0 & 1 & -1 & 1 & -3\\
                0 & 5 & -3 & 1 & -3\\
                0 & -7 & 3 & 1 & -3 
                \end{array}
            \right)
            \xrightarrow[7r_2\rightarrow r_4]{-5r_2 \rightarrow r_3,2r_2 \rightarrow r_1}
            \]
            \[
            \left(\begin{array}{ccccc}
                1 & 0 & 1 & -2 & -2\\
                0 & 1 & -1 & 1 & -3\\
                0 & 0 & 2 & -4 & 12\\
                0 & 0 & -4 & 8 & -24 
                \end{array}
            \right)
            \xrightarrow[\frac{1}{2} r_3]{\frac{1}{4} r_4}
            \left(\begin{array}{ccccc}
                1 & 0 & 1 & -2 & -2\\
                0 & 1 & -1 & 1 & -3\\
                0 & 0 & 1 & -2 & 6\\
                0 & 0 & -1 & 2 & -6
                \end{array}
            \right)
            \xrightarrow[r_3 \rightarrow r_4,r_3 \rightarrow r_2]{-r_3 \rightarrow r_1}
            \left(\begin{array}{ccccc}
                1 & 0 & 0 & 0 & -8\\
                0 & 1 & 0 & -1 & 3\\
                0 & 0 & 1 & -2 & 6\\
                0 & 0 & 0 & 0 & -0
                \end{array}
            \right)
            \]
            即
            \[
                \begin{cases}
                    \ x_1&=-8,\\
                    \ x_2-x_4&=3,\\
                    \ x_3-2x_4&=6.
                \end{cases}
                \ \Rightarrow\ 
                \begin{cases}
                    \ x_1&=-8,\\
                    \ x_2&=x_4+3,\\
                    \ x_3&=2x_4+6,\\
                    \ x_4&=x_4.
                \end{cases}
            \]
            令$x_4=t$,解得
            \[
                \left(
                    \begin{array}{ccc}
                        x_1\\
                        x_2\\
                        x_3\\
                        x_4
                    \end{array}
                \right)
                =
                \left(
                    \begin{array}{ccc}
                        0\\
                        1\\
                        2\\
                        1
                    \end{array}
                \right)
                t+
                \left(
                    \begin{array}{ccc}
                        -8\\
                        3\\
                        6\\
                        0
                    \end{array}
                \right)
            \]
            \item [(4)]
            \[
                \begin{cases}
                    \ 2x_1 +4x_2 -6x_3 +x_4&=2,\\
                    \ \ \,x_1 -\ x_2 +4 \, x_3 +x_4&=1,\\
                    \ -x_1+x_2-\ \ x_3+x_4&=0.
                \end{cases}
            \]

            $
            \begin{pmatrix}
                2& 4& -6& 1& 2\\
                1& -1& 4& 1& 1\\
                -1& 1& -1& 1& 0
            \end{pmatrix}
            \xrightarrow[-2r_1 \rightarrow r_1]{r2 \rightarrow r_3}
            \begin{pmatrix}
                0& 6& -14& -1& 0\\
                1& -1& 4& 1& 1\\
                0& 0& 3& 2& 1
            \end{pmatrix}
            \xrightarrow{r_1 \leftrightarrow r_2}
            \begin{pmatrix}
                1& -1& 4& 1& 1\\
                0& 6& -14& -1& 0\\
                0& 0& 3& 2& 1
            \end{pmatrix}
            $

            \[
            \xrightarrow[\frac{-4}{3}r_3 \rightarrow r_1]{\frac{14}{3}r_3 \rightarrow r_2}
            \left(
                    \begin{array}{ccccc}
                        1& -1& 0& -5/3& -1/3\\
                        0& 6& 0& 25/3& 14/3\\
                        0& 0& 3& 2& 1
                    \end{array}
                \right)
            \xrightarrow{\frac{1}{6}r_2 \rightarrow r_1}
            \left(
                    \begin{array}{ccccc}
                        1& 0& 0& -5/18& 4/9\\
                        0& 6& 0& 25/3& 14/3\\
                        0& 0& 3& 2& 1
                    \end{array}
                \right)
            \]

            即:
            \[
                \begin{cases}
                    \ x_1 -\frac{5}{18}x_4 &=\frac{4}{9},\\
                    \ 6x_2 +\frac{25}{3}x_4 &=\frac{14}{3},\\
                    \ 3x_3 +2x_4 &=1.
                \end{cases}
                \quad \Rightarrow
                \begin{cases}
                    \ x_1&=\frac{5}{18}x_4+\frac{4}{9},\\
                    \ x_2&=\frac{-25}{18}x_4+\frac{7}{9},\\
                    \ x_3&=\frac{-2}{3}x_4+\frac{1}{3},\\
                    \ x_4&=x_4,
                \end{cases}
            \]
            令$x_4=t$,即
            \[
                \left(
                    \begin{array}{ccc}
                        x_1\\
                        x_2\\
                        x_3\\
                        x_4
                    \end{array}
                \right)
                =
                \left(\begin{array}{ccc}
                        5/18\\
                        -25/18\\
                        -2/3\\
                        1
                    \end{array}\right)
                t+
                \left(\begin{array}{ccc}
                        4/9\\
                        7/9\\
                        1/3\\
                        0
                    \end{array}\right)
            \]
    
            \item [(6)]
            \[
                \begin{cases}
                    \ \ 3x_1-\ 5x_2+\ x_3-2x_4&=0.\\
                    \ \ 2x_1+\ 3x_2-5x_3+\ x_4&=0,\\
                    \ -x_1+\ 7x_2-4x_3+3x_4&=0,\\
                    \ \ 4x_1+15x_2-7x_3+9x_4&=0.
                \end{cases}
            \]
            $
            \begin{pmatrix}
                3& -5& 1& -2& 0\\
                2& 3& -5& 1& 0\\
                -1& 7& -4& 3& 0\\
                4& 15& -7& 9& 0
            \end{pmatrix}
            \xrightarrow[4r_3 \rightarrow r_4]{3r_3 \rightarrow r_1,2r_3 \rightarrow r_2}
            \begin{pmatrix}
                0& 16& -11& 7& 0\\
                0& 17& -13& 7& 0\\
                -1& 7& -4& 3& 0\\
                0& 43& -23& 21& 0
            \end{pmatrix}
            \xrightarrow[r_1 \leftrightarrow r_3]{-r_1 \rightarrow r_2,-3r_1 \rightarrow r_4}
            $

            $
                \begin{pmatrix}
                    -1& 7& -4& 3& 0\\
                    0& 1& -2& 0& 0\\
                    0& 16& -11& 7& 0\\
                    0& -5& 10& 0& 0
                \end{pmatrix}
                \xrightarrow[-16r_2 \rightarrow r_3]{5x_2 \rightarrow x_4,-7r_2 \rightarrow r_1}
                \begin{pmatrix}
                    -1& 0& 10& 3& 0\\
                    0& 1& -2& 0& 0\\
                    0& 0& 21& 7& 0\\
                    0& 0& 0& 0& 0    
                \end{pmatrix}
                \xrightarrow[-3x_3 \rightarrow x_1]{\frac{1}{7}x_3}
                \begin{pmatrix}
                    -1& 0& 1& 0& 0\\
                    0& 1& -2& 0& 0\\
                    0& 0& 3& 1& 0\\
                    0& 0& 0& 0& 0 
                \end{pmatrix}
            $

            即
            \[
                \begin{cases}
                    \ -x_1+x_3&=0,\\
                    \ x_2-2x_3&=0,\\
                    \ 3x_3+x_4&=0.\\
                \end{cases}
                \quad \Rightarrow
                \begin{cases}
                    \ x_1&=x_3,\\
                    \ x_2&=2x_3,\\
                    \ x_4&=-3x_3.
                \end{cases}
            \]
            令$x_3=t$,即
            \[
                \left(\begin{array}{ccc}
                        x_1\\
                        x_2\\
                        x_3\\
                        x_4
                    \end{array}\right)
                =
                \left(\begin{array}{ccc}
                        1\\
                        2\\
                        1\\
                        -3
                    \end{array}\right)
                t
            \]
        \end{enumerate}
        \item [2.]当a为何值时,下列线性方程组有解?有解时求出它的通解:
        \begin{enumerate}
            \item [(1)]
            \[
                \begin{cases}
                    \ 3x_1+2x_2+x_3&=2,\\
                    \ \ \,x_1-\ x_2-2x_3&=-3,\\
                    \ ax_1-2x_2+2x_3&=6;
                \end{cases}
            \]
            $
            \begin{pmatrix}
                3& 2& 1& 2\\
                1& -1& -2& -3\\
                a& -2& 2& 6
            \end{pmatrix}
            \xrightarrow[-ax_2 \rightarrow r_3]{-3x_2 \rightarrow x_1}
            \begin{pmatrix}
                0& 5& 7& 11\\
                1& -1& -2& -3\\
                0& a-2& 2a+2& 3a+6
            \end{pmatrix}
            \xrightarrow[r_1 \leftrightarrow r_2]{\frac{1}{5}r_1 \rightarrow r_2,\frac{2-a}{5}x_1 \rightarrow x_3}
            $

            $
                \begin{pmatrix}
                    1& 0& -3/5& -4/5\\
                    0& 5& 7& 11\\
                    0& 0& (3a+24)/5& (4a+52)/5
                \end{pmatrix}
                \xrightarrow[5x_1]{5x_3}
                \begin{pmatrix}
                    5& 0& -3& -4\\
                    0& 5& 7& 11\\
                    0& 0& 3(a+8)& 4(a+13)
                \end{pmatrix}
            $

            即解:
            \[
                \begin{cases}
                    \ 5x_1-3x_3&=-4,\\
                    \ 5x_2+7x_3&=11,\\
                    \ 3(a+8)x_3&=4(a+13).
                \end{cases}
            ,\ \mbox{当a+8 $\neq $0,即a $\neq$ -8时有解}
                \begin{cases}
                    \ x_1&=4/(a+8),\\
                    \ x_2&=(20-a)/(3a+24),\\
                    \ x_3&=(4a+52)/(3a+24).
                \end{cases}
            \]

            即a$\neq $-8时有解,通解为$\displaystyle{\left(\frac{4}{a+8} ,\frac{20-a}{3(a+8)} ,\frac{4(a+13)}{3(a+8)}\right)}$
        \end{enumerate}
        \item [3.]a为何值时,下述线性方程组有唯一解?a为何值时,此方程组无解?
        \[
        \begin{cases}
            \ x_1 +x_2 +x_3&=3,\\
            \ x_1 +2x_2 -ax_3&=9,\\
            \ 2x_1 -x_2 +3x_3&=6.
        \end{cases}
        \]
        $
        \begin{pmatrix}
            1& 1& 1& 3\\
            1& 2& -a& 9\\
            2& -1& 3& 6
        \end{pmatrix}
        \xrightarrow[-2r_1 \rightarrow r_3]{-r_1 \rightarrow r_2}
        \begin{pmatrix}
            1& 1& 1& 3\\
            0& 1& -a-1& 6\\
            0& -3& 1& 0
        \end{pmatrix}
        \xrightarrow[3r_2 \rightarrow r_3]{-r_2 \rightarrow r_1}
        \begin{pmatrix}
            1& 0& a+2& -3\\
            0& 1& -a-1& 6\\
            0& 0& -3a-2& 18
        \end{pmatrix}
        $

        即解:
        \[
            \begin{cases}
                \ x_1+(a+2)x_3&=-3,\\
                \ x_2-(a+1)x_3&=6,\\
                \ -(3a+2)x_3&=18.
            \end{cases}
            ,\ \mbox{当$3a+2 \neq 0$,即$a \neq \frac{-2}{3}$时有解,且为唯一解.}
        \]
        即$a \neq \frac{-2}{3}$时有唯一解,$a=\frac{-2}{3}$时无解.
        
        \item [5.]求三次多项式$f(x)=ax^3+bx^2+cx+d$满足:
        \[f(0)=1,f(1)=2,f'(0)=1,f'(1)=-1.\]
        代入,得:
        \[
            \begin{cases}
                \ \qquad \qquad \qquad d&=1\\
                \ \ a+\ b+\,c+d&=2\\
                \ \qquad \qquad \;c&=1\\
                \ 3a+2b+c&=-1
            \end{cases}
        \]
        $
        \begin{pmatrix}
            0& 0& 0& 1& 1\\
            1& 1& 1& 1& 2\\
            0& 0& 1& 0& 1\\
            3& 2& 1& 0& -1
        \end{pmatrix}
        \xrightarrow[-r_3 \rightarrow r_2,-3r_2 \rightarrow r_4]{-r_1 \rightarrow r_2,-r_3 \rightarrow r_4}
        \begin{pmatrix}
            0& 0& 0& 1& 1\\
            1& 1& 0& 0& 0\\
            0& 0& 1& 0& 1\\
            0& -1& 0& 0& -2
        \end{pmatrix}
        \xrightarrow[r_2\leftrightarrow r_4]{r_1 \leftrightarrow r_2}
        \begin{pmatrix}
            1& 1& 0& 0& 0\\
            0& -1& 0& 0& -2\\
            0& 0& 1& 0& 1\\
            0& 0& 0& 1& 1
        \end{pmatrix}
        $

        即解:
        \[
            \begin{cases}
                a+b&=0\\
                -b&=-2\\
                c&=1\\
                d&=1
            \end{cases}
            \Rightarrow
            \begin{cases}
                a&=-2\\
                b&=2\\
                c&=1\\
                d&=1
            \end{cases}
        \]
        即$f(x)=-2x^3+2x^2+x+1$.
        \item [7.]给定线性方程组
        \[
            \begin{cases}
                \ \ x_1+2x_2-3x_3+4x_4&=2,\\
                \ 2x_1+5x_2-2x_3+\;x_4&=1,\\
                \ 3x_1+8x_2-\ x_3-2x_4&=0.
            \end{cases}
        \]
        将常数项改为零得到另一个方程组,求解这两个方程组,并研究这两个方程组的解之问的关系,对其他方程组做类似的讨论.
        
        解:
        \[
            \begin{cases}
                \ \ x_1+2x_2-3x_3+4x_4&=a,\\
                \ 2x_1+5x_2-2x_3+\;x_4&=b,\\
                \ 3x_1+8x_2-\ x_3-2x_4&=c.
            \end{cases}
        \]

        \[
            \begin{pmatrix}
                1& 2& -3& 4& a\\
                2& 5& -2& 1& b\\
                3& 8& -1& -2& c
            \end{pmatrix}
            \xrightarrow[-3r_1 \rightarrow r_3]{-2r_1 \rightarrow r_2}
            \begin{pmatrix}
                1& 2& -3& 4& a\\
                0& 1& 4& -7& b-2a\\
                0& 2& 8& -14& c-3a
            \end{pmatrix}
        \]
        
        \[
            \xrightarrow[-2r_2 \rightarrow r_3]{-2r_2 \rightarrow r_1}
            \begin{pmatrix}
            1& 0& -11& 18& 5a-2b\\
            0& 1& 4& -7& b-2a\\
            0& 0& 0& 0& a-2b+c
            \end{pmatrix}
        \]
        
        齐次方程 (a=b=c=0):
        \[
            \begin{cases}
                \ x_1\quad -11x_3+18x_4&=0,\\
                \ \quad \ x_2+\ 4x_3-\ 7x_4&=0.
            \end{cases}
            \xrightarrow[x_4=n]{x_3=m}
            \begin{cases}
                \ x_1&=11m-18n,\\
                \ x_2&=-4m+7n,\\
                \ x_3&=m,\\
                \ x_4&=n.
            \end{cases}
            \]
            \[
            \Rightarrow
            \left(
                \begin{array}{ccc}
                    x_1\\
                    x_2\\
                    x_3\\
                    x_4
                \end{array}
            \right)
            =
            \left(\begin{array}{ccc}
                    11\\
                    -4\\
                    1\\
                    0
                \end{array}\right)
            m+
            \left(\begin{array}{ccc}
                    -18\\
                    7\\
                    0\\
                    1
                \end{array}\right)
            n
        \]
        非齐次方程 (a=2,b=1,c=0):
        \[
            \begin{cases}
                \ x_1\quad -11x_3+18x_4&=8,\\
                \ \quad \ x_2+\ 4x_3-\ 7x_4&=-3.
            \end{cases}
            \xrightarrow[x_4=n]{x_3=m}
            \begin{cases}
                \ x_1&=11m-18n+8,\\
                \ x_2&=-4m+7n-3,\\
                \ x_3&=m,\\
                \ x_4&=n.
            \end{cases}
            \]
            \[
            \Rightarrow
            \left(
                \begin{array}{ccc}
                    x_1\\
                    x_2\\
                    x_3\\
                    x_4
                \end{array}
            \right)
            =
            \left(
                \begin{array}{ccc}
                    11\\
                    -4\\
                    1\\
                    0
                \end{array}
            \right)
            m+
            \left(\begin{array}{ccc}
                    -18\\
                    7\\
                    0\\
                    1
                \end{array}\right)
            n+
            \left(\begin{array}{ccc}
                    8\\
                    -3\\
                    0\\
                    0
                \end{array}\right)
        \]
        非齐次方程的通解比对应的齐次方程多一组常量.
    \end{enumerate}
    \subsection{习题四}
    \begin{enumerate}
        \item [2.]证明:每个方阵都可以表示为一个对称矩阵和一个反对称矩阵之和的形式.
        
        $\forall A={(a_{ij})}_{n*n}$,假设$\exists B=(b_{ij}),C=(c_{ij})$,$B$为对称矩阵,$C$为反对称矩阵,且$B+C=A$,即:
        \[
            \begin{cases}
                \ b_{ij}+c_{ij}&=a_{ij};\\
                \ b_{ij}-c_{ij}&=a_{ji}.
            \end{cases}
            \Rightarrow
            \begin{cases}
                \ b_{ij}&=\displaystyle{\frac{a_{ij}+a_{ji}}{2}};\\
                \ c_{ij}&=\displaystyle{\frac{a_{ij}-a_{ji}}{2}}.
            \end{cases}
            \ \mbox{满足}
            \begin{cases}
                \ b_{ij}&=b_{ji};\\
                \ c_{ij}&=-c_{ji}.
            \end{cases}
        \]
        即$A=\frac{1}{2}(A+A^T)+\frac{1}{2}(A-A^T)$.

        即证每个方阵都可以表示为一个对称矩阵和一个反对称矩阵之和的形式.
        \item [3.]设
        $A=\begin{pmatrix}
            -3& -1& -2\\
            1& 3& 4 
        \end{pmatrix},
        B=\begin{pmatrix}
            2& 2& -2\\
            4& -1& -4\\
            4& 3& -3
        \end{pmatrix},
        C=\begin{pmatrix}
            1& 1\\
            -4& 1\\
            -1& -2
        \end{pmatrix}.
        $
        计算$AB,BC,ABC,B^2,AC,CA$.
        \begin{enumerate}
            \item [(1)]
            \[
                AB=
                \begin{pmatrix}
                    -3& -1& -2\\
                    1& 3& 4
                \end{pmatrix}
                \begin{pmatrix}
                    2& 2& -2\\
                    4& -1& -4\\
                    4& 3& -3
                \end{pmatrix}
                =
                \begin{pmatrix}
                    -18& -11& 16\\
                    30& 11& -26
                \end{pmatrix}
            \]
            \item [(2)]
            \[
                BC=
                \begin{pmatrix}
                    2& 2& -2\\
                    4& -1& -4\\
                    4& 3& -3
                \end{pmatrix}
                \begin{pmatrix}
                    1& 1\\
                    -4& 1\\
                    -1& -2
                \end{pmatrix}
                =
                \begin{pmatrix}
                    -4& 8\\
                    12& 11\\
                    -5& 13
                \end{pmatrix}
            \]
            \item [(3)]
            \begin{align*}
                ABC&=
                \begin{pmatrix}
                    -3& -1& -2\\
                    1& 3& 4 
                \end{pmatrix}
                \begin{pmatrix}
                    2& 2& -2\\
                    4& -1& -4\\
                    4& 3& -3
                \end{pmatrix}
                \begin{pmatrix}
                    1& 1\\
                    -4& 1\\
                    -1& -2
                \end{pmatrix}\\
                &=
                \begin{pmatrix}
                    -18& -11& 16\\
                    30& 11& -26
                \end{pmatrix}
                \begin{pmatrix}
                    1& 1\\
                    -4& 1\\
                    -1& -2
                \end{pmatrix}\\
                &=
                \begin{pmatrix}
                    10& -61\\
                    12& 93
                \end{pmatrix}
            \end{align*}
            \item [(4)]
            \[
                B^2=
                \begin{pmatrix}
                    2& 2& -2\\
                    4& -1& -4\\
                    4& 3& -3
                \end{pmatrix}
                \begin{pmatrix}
                    2& 2& -2\\
                    4& -1& -4\\
                    4& 3& -3
                \end{pmatrix}
                =
                \begin{pmatrix}
                    4& -4& -6\\
                    -12& -3& 8\\
                    8& -4& -11
                \end{pmatrix}
            \]
            \item [(5)]
            \[
                AC=
                \begin{pmatrix}
                    -3& -1& -2\\
                    1& 3& 4 
                \end{pmatrix}
                \begin{pmatrix}
                    1& 1\\
                    -4& 1\\
                    -1& -2
                \end{pmatrix}
                =
                \begin{pmatrix}
                    3& 0\\
                    -15& -4
                \end{pmatrix}
            \]
            \item [(6)]
            \[
                CA=
                \begin{pmatrix}
                    1& 1\\
                    -4& 1\\
                    -1& -2
                \end{pmatrix}
                \begin{pmatrix}
                    -3& -1& -2\\
                    1& 3& 4 
                \end{pmatrix}
                =
                \begin{pmatrix}
                    -2& 2& 2\\
                    3& 7& 12\\
                    1& -5& -6
                \end{pmatrix}
            \]
        \end{enumerate}
        \item [5.]计算
        $\begin{pmatrix}
            x_1& x_2& \cdots & x_n
        \end{pmatrix}
        \begin{pmatrix}
            a_{11} & a_{12} & \cdots & a_{1n}\\
            a_{21} & a_{22} & \cdots & a_{2n}\\
            \vdots & \vdots & \ddots & \vdots \\
            a_{n1} & a_{n2} & \cdots & a_{nn}
        \end{pmatrix}
        \begin{pmatrix}
            y_1\\
            y_2\\
            \vdots\\
            y_n
        \end{pmatrix}.
        $

        \begin{align*}
            \mbox{原式}
            &=\left(
            \begin{array}{ccccc}
                \sum\limits_{i=1}^{n} x_{i} a_{i1} & \sum\limits_{i=1}^{n} x_{i} a_{i2} &\cdots &\sum\limits_{i=1}^{n} x_{i} a_{in}
            \end{array}
            \right)
            \begin{pmatrix}
                y_1\\
                y_2\\
                \vdots\\
                y_n
            \end{pmatrix}\\
            &=\displaystyle{\sum\limits_{i=1}^{n} \sum\limits_{j=1}^{n} x_i y_j a_{ij}}.
        \end{align*}
        \item [6.]举出满足下列条件的2阶实方阵$A$:
        \begin{enumerate}
            \item [(1)]
            $A^2=
            \begin{pmatrix}
                0 & 1\\
                1 & 0
            \end{pmatrix}$;

            $
            \begin{pmatrix}
                a & b\\
                c & d
            \end{pmatrix}^2
            =
            \begin{pmatrix}
                a^2+bc & ab+bd\\
                ac+cd & bc+d^2
            \end{pmatrix}
            =
            \begin{pmatrix}
                0 & 1\\
                1 & 0
            \end{pmatrix}
            \Rightarrow
            \begin{cases}
                \ a^2+bc=d^2+bc&=0\\
                \ ab+bd=ac+cd&=1
            \end{cases}
            $
            \begin{enumerate}
                \item [$1^\circ $]$(a=d\neq 0)$:
                
                \[b(a+d)=c(a+d)=1 \Rightarrow b=c=\frac{1}{a+d}\neq 0 \Rightarrow a^2=d^2=-bc<0.\] 矛盾.
                \item [$2^\circ $]$(a=-d)$:
                
                \[b(a+d)=c(a+d)=0\neq 1.\]矛盾. 
            \end{enumerate}
            即不存在满足此条件的方阵$A$.
            \item [(2)]
            $A^2=
            \begin{pmatrix}
                0 & 1\\
                -1 & 0
            \end{pmatrix}$;

            \[
                A:
                \begin{pmatrix}
                    \frac{\sqrt{2}}{2} & \frac{\sqrt{2}}{2}\\
                    \frac{-\sqrt{2}}{2} & \frac{\sqrt{2}}{2}
                \end{pmatrix}
            \]
            \item [(3)]
            $A^3=I$且$A \neq I$.

            \[
                A:
                \begin{pmatrix}
                    \frac{-1}{2} & \frac{\sqrt{3}}{2}\\
                    \frac{-\sqrt{3}}{2} & \frac{-1}{2}
                \end{pmatrix}
            \]
        \end{enumerate}
        \item [8.]设$A,B$都是n阶对称方阵,且AB=BA.证明:AB也是对称方阵.
        
        \[{(AB)}^{T}=B^{T} A^{T}=BA=AB\]即证.
        \item [9.]证明:两个$n$阶上 (下)三角方阵的乘积仍是上 (下)三角方阵.
        \begin{enumerate}
            \item [(1)]
            设$A=(a_{ij}),B=(b_{ij})$为$n$阶上三角方阵,则$a_{ij}=b_{ij}=0,\ \forall 1\leq j<i \leq n$.

            设$AB=C=(c_{ij})$则$c_{ij}=\displaystyle{\sum\limits_{k=1}^{n} a_{ik}b_{kj}}$,考察$1\leq j<i \leq n$时$c_{ij}$的取值情况.
            \[
                \begin{cases}
                    \ a_{ik}&=0 \quad 1\leq k \leq j,\\
                    \ b_{kj}&=0 \quad j < k \leq n.
                \end{cases}
                \quad \Rightarrow
                \begin{cases}
                    \ a_{ik}b_{kj}&=0 \quad 1\leq k \leq j,\\
                    \ a_{ik}b_{kj}&=0 \quad j < k \leq n.
                \end{cases}
                \quad \Rightarrow
                \forall 1\leq j<i \leq n,c_{ij}=0.
            \]
            即证两个$n$阶上三角方阵的乘积仍是上三角方阵.
            \item [(2)]
            设$A=(a_{ij}),B=(b_{ij})$为$n$阶下三角方阵,则$a_{ij}=b_{ij}=0,\ \forall 1\leq i<j \leq n$.

            设$AB=C=(c_{ij})$则$c_{ij}=\displaystyle{\sum\limits_{k=1}^{n} a_{ik}b_{kj}}$,,考察$1\leq i<j \leq n$时$c_{ij}$的取值情况.
            \[
                \begin{cases}
                    \  b_{kj}&=0 \quad 1\leq k \leq j,\\
                    \  a_{ik}&=0 \quad j < k \leq n.
                \end{cases}
                \quad \Rightarrow
                \begin{cases}
                    \ a_{ik}b_{kj}&=0 \quad 1\leq k \leq j,\\
                    \ a_{ik}b_{kj}&=0 \quad j < k \leq n.
                \end{cases}
                \quad \Rightarrow
                \forall 1\leq i<j \leq n,c_{ij}=0.
            \]
            即证两个$n$阶下三角方阵的乘积仍是下三角方阵.
        \end{enumerate}
    \end{enumerate}
\end{document}