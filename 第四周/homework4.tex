\documentclass{article}
\usepackage{xeCJK,amsmath,geometry,graphicx,amssymb,zhnumber,booktabs,setspace,tasks,verbatim,amsthm,amsfonts,mathdots}
\geometry{a4paper,scale=0.8}   
\title{线性代数homework (第三周)}
\author{PB20000113孔浩宇}
\date{\today}
\begin{document}
\maketitle
\section{周二}
\subsection{习题四 23}
\begin{enumerate}
    \item [23.]计算下列行列式:
    \begin{enumerate}
        \item [(1)]
        \[
            \begin{vmatrix}
                1 & 0 & 1 & -4\\
                -1& -3& -4& -2\\
                2 & -1& 4 & 4 \\
                2 & 3 & -3& 2
            \end{vmatrix}=
            \begin{vmatrix}
                1 & 0 & 0 & 0\\
                -1& -3& -3& -6\\
                2 & -1& 2 & 12\\
                2 & 3 & -5& 10
            \end{vmatrix}=
            \begin{vmatrix}
                1 & 0 & 0 & 0\\
                -1& -3& 0& 0\\
                2 & -1& 3 & 14\\
                2 & 3 & -8& 4
            \end{vmatrix}=-372.
        \]
        \item [(2)]
            \[
            \begin{vmatrix}
                1& 4& -1& -1\\
                1& -2& -1& 1\\
                -3& 3& -4& -2\\
                0& 1& -1& -1
            \end{vmatrix}=
            \begin{vmatrix}
                1& 0 & 0 & 0\\
                1& -6& 0 & 2\\
                -3& 15& -7& -5\\
                0& 1 & -1& -1
            \end{vmatrix}=
            \begin{vmatrix}
                1&  0 & 0 & 0\\
                1&  0 & 0 & 2\\
                -3& 0 & -7& -5\\
                0&  -2 & -1& -1
            \end{vmatrix}=
            \begin{vmatrix}
                0& 0& 2\\
                0& -7& -5\\
                -2& -1& -1
            \end{vmatrix}
            =14.
            \]       
        \item [(3)]
        \[
            \begin{vmatrix}
                x+a & x+b & x+c\\
                y+a & y+b & y+c\\
                z+a & z+b & z+c
            \end{vmatrix}=
            \begin{vmatrix}
                x+a & x+b & x+c\\
                y-x & y-x & y-x\\
                z-x & z-x & z-x
            \end{vmatrix}=
            \begin{vmatrix}
                x+a & b-a & c-a\\
                y-x & 0 & 0\\
                z-x & 0 & 0
            \end{vmatrix}=0.
        \]
        \item [(7)]
        \[
            \mbox{记}D_n=
            \begin{vmatrix}
                a_1 &        &     &     &        &b_1\\
                    & \ddots &     &     & \iddots &   \\
                    &        & a_n & b_n &        &   \\
                    &        & c_n & d_n &        &   \\
                    & \iddots &     &     & \ddots &   \\
                c_1 &        &     &     &        &d_1
            \end{vmatrix}
            \Rightarrow
            \begin{cases}
                D_1&=a_1 d_1 - b_1 c_1,\\
                D_2&=(a_1 d_1 - b_1 c_1)(a_2 d_2 - b_2 c_2)       
           \end{cases}.
        \]
        假设$D_n=\prod\limits_{i=1}^{n}(a_i d_i - b_i c_i)$.
        \begin{enumerate}
            \item [$(a)$]$n=1$成立.
            \item [$(b)$]假设$D_n=\prod\limits_{i=1}^{n}(a_i d_i - b_i c_i)$对$n=k$成立,则
            \[
                D_{k+1}=(a_1 d_1 - b_1 c_1)
                \begin{vmatrix}
                    a_2 &         &         &         &         & b_2\\
                        & \ddots  &         &         & \iddots &    \\
                        &         & a_{k+1} & b_{k+1} &         &    \\
                        &         & c_{k+1} & d_{k+1} &         &    \\
                        & \iddots &         &         & \ddots  &    \\
                    c_2 &         &         &         &         & d_2
                \end{vmatrix}
                =\prod\limits_{i=1}^{k+1}(a_i d_i - b_i c_i)
            \]
            $n=k+1$时原式仍成立.
        \end{enumerate}
        综合$(a)(b)$即证$D_n=\prod\limits_{i=1}^{n}(a_i d_i - b_i c_i)$对任意$n\in \mathbb{N}^*$成立.
        
        即$D_n=\prod\limits_{i=1}^{n}(a_i d_i - b_i c_i)$.
    \end{enumerate}
\end{enumerate}
\subsection{补充题}
\begin{enumerate}
    \item [1.]对于$a,b,c,d$为不全为$0$的实数
    \[
        \begin{pmatrix}
            a & b & c & d\\
            d & a & b & c\\
            c & d & a & b\\
            b & c & d & a
        \end{pmatrix}
    \]
    求$M$的行列式.

    做多项式$f(x)=a+bx+cx^2+dx^3$,令$\omega_{k}=\frac{2k\pi i}{4}=\frac{k\pi i}{2}$.
    \begin{align*}
        \begin{vmatrix}
            a & b & c & d\\
            d & a & b & c\\
            c & d & a & b\\
            b & c & d & a
        \end{vmatrix}
        \begin{vmatrix}
            1 & 1 & 1 & 1\\
            \omega_1 & \omega_1^2 & \omega_1^3 & \omega_1^4\\
            \omega_2 & \omega_2^2 & \omega_2^3 & \omega_2^4\\
            \omega_3 & \omega_3^2 & \omega_3^3 & \omega_3^4
        \end{vmatrix}
        &=
        \det
        \begin{pmatrix}
            a & b & c & d\\
            d & a & b & c\\
            c & d & a & b\\
            b & c & d & a
        \end{pmatrix}
        \begin{pmatrix}
            1 & 1 & 1 & 1\\
            \omega_1 & \omega_2 & \omega_3 & \omega_4\\
            \omega_1^2 & \omega_2^2 & \omega_3^2 & \omega_4^2\\
            \omega_1^3 & \omega_2^3 & \omega_3^3 & \omega_4^3
        \end{pmatrix}\\
        &=
        \det
        \begin{pmatrix}
            f(\omega_1) & f(\omega_2) & f(\omega_3) & f(\omega_4)\\
            \omega_1 f(\omega_1) & \omega_2 f(\omega_2) & \omega_3 f(\omega_3) & \omega_4 f(\omega_4)\\
            \omega_1^2 f(\omega_1) & \omega_2^2 f(\omega_2) & \omega_3^2 f(\omega_3) & \omega_4^2 f(\omega_4)\\
            \omega_1^3 f(\omega_1) & \omega_2^3 f(\omega_2) & \omega_3^3 f(\omega_3) & \omega_4^3 f(\omega_4)
        \end{pmatrix}\\
        &=
        f(\omega_1) f(\omega_2) f(\omega_3) f(\omega_4)
        \begin{vmatrix}
            1 & 1 & 1 & 1\\
            \omega_1 & \omega_1^2 & \omega_1^3 & \omega_1^4\\
            \omega_2 & \omega_2^2 & \omega_2^3 & \omega_2^4\\
            \omega_3 & \omega_3^2 & \omega_3^3 & \omega_3^4
        \end{vmatrix}
    \end{align*}
    又$\omega_k$各不相同,
    $\begin{vmatrix}
        1 & 1 & 1 & 1\\
        \omega_1 & \omega_2 & \omega_3 & \omega_4\\
        \omega_1^2 & \omega_2^2 & \omega_3^2 & \omega_4^2\\
        \omega_1^3 & \omega_2^3 & \omega_3^3 & \omega_4^3
    \end{vmatrix}
    \neq 0,\Rightarrow
    \begin{vmatrix}
        a & b & c & d\\
        d & a & b & c\\
        c & d & a & b\\
        b & c & d & a
    \end{vmatrix}=f(\omega_1) f(\omega_2) f(\omega_3) f(\omega_4)
    $.

    即:
    \begin{align*}
        \begin{vmatrix}
            a & b & c & d\\
            d & a & b & c\\
            c & d & a & b\\
            b & c & d & a
        \end{vmatrix}
        &=f(\omega_1) f(\omega_2) f(\omega_3) f(\omega_4)\\
        &=(a+b+c+d)(a-b+c-d)(a+bi-c-di)(a-bi-c+di)\\
        &=[{(a+c)}^2-{(b+d)}^2][{(a-c)}^2+{(b-d)}^2].
    \end{align*}
    \item [2.]写出行列式$M$中含有$x^4$和$x^3$的项.
    \[
        M=
        \begin{vmatrix}
            x & 1 & 2 & 3\\
            x & x & 1 & 2\\
            2 & 3 & x & 1\\
            x & 2 & 3 & x
        \end{vmatrix}.
    \]
    \[
        M=
        x\cdot
        \begin{vmatrix}
            x & 1 & 2\\
            3 & x & 1\\
            2 & 3 & x
        \end{vmatrix} 
        -x\cdot
        \begin{vmatrix}
            1 & 2 & 3\\
            3 & x & 1\\
            2 & 3 & x
        \end{vmatrix} 
        +2\cdot
        \begin{vmatrix}
            1 & 2 & 3\\
            x & 1 & 2\\
            2 & 3 & x
        \end{vmatrix} 
        -x\cdot
        \begin{vmatrix}
            1 & 2 & 3\\
            x & 1 & 2\\
            3 & x & 1
        \end{vmatrix}
    \]
    \begin{enumerate}       
        \item [(1)]$x^4$:仅第一部分有$x^4$,易得$M$中含有$x^4$的项为$x^4$.
        \item [(2)]$x^3$:仅第二、四部分有$x^3$,易得$M$中含有$x^3$的项为$-x^3-3x^3=-4x^3$.
    \end{enumerate}
    \item [3.]计算行列式 (参考第四题)
    \[
        M=
        \begin{vmatrix}
            1  & 0  & 1 & 0 & 0\\
            -3 & 1  & 3 & 1 & 0\\
            2  & -3 & 2 & 3 & 1\\
            0  & -2 & 0 & 2 & 3\\
            0  &  0 & 0 & 0 & 2
        \end{vmatrix}
    \]
    记$f(x)=x^3-3x^2+2x,g(x)=x^2+3x+2$.
    \[
        \mbox{解}f(x)=0
        \Rightarrow
        \begin{cases}
            x_1&=0,\\
            x_2&=1,\\
            x_3&=2.
        \end{cases}        
        \Rightarrow
        M=1^2\cdot g(x_1)\cdot g(x_2)\cdot g(x_3)=144.
    \]
    \item [4.]对于多项式$A,B\in \mathbb{C}[x]$有如下的展开
    \begin{align*}
        A&=a_0 x^d +a_1 x^{d-1} +\cdots +a_d\\
        B&=b_0 x^e +b_1 x^{e-1} +\cdots +b_e
    \end{align*}
    并且它们的根分别为$\lambda_1,\ldots,\lambda_d$与$\mu_1,\ldots,\mu_e$.我们定义它们的Sylvester行列式为
    \[
        M=
        \begin{pmatrix}
            a_0    &   0    & \cdots &      0 &    b_0 &      0 & \cdots & 0\\
            a_1    & a_0    & \cdots &      0 &    b_1 &    b_0 & \cdots & 0\\
            a_2    & a_1    & \ddots &      0 &    b_2 &    b_1 & \ddots & 0\\
            \vdots & \vdots & \ddots &    a_0 & \vdots & \vdots & \ddots & b_0\\
            a_d    & a_{d-1}& \cdots & \vdots &    b_e & b_{e-1}& \cdots & \vdots\\
            0      & a_d    & \ddots & \vdots &      0 &    b_e & \ddots & \vdots\\
            \vdots & \vdots & \ddots & a_{d-1}& \vdots & \vdots & \ddots & b_{e-1}\\
            0      &   0    & \cdots &    a_d &      0 &      0 & \cdots & b_e
        \end{pmatrix}
    \]
    这对多项式的结式 (Resultant)定义为他们的Sylvester多项式的行列式,即\[res(A, B) = \det M.\]
    请验证有如下的关系:\[res (A, B)={a_0}^e {b_0}^d \prod\limits_{1\leq i\leq d \atop 1\leq j\leq e}(\lambda_i -\mu_j)\]
    \begin{enumerate}
        \item [Remarque.]根据结果我们知道,结式可以用来判断两个多项式是否有公共的根,进而可以
        知道两个多项式是否互素。除此以外,结式在数论、代数几何、交换代数中都有广泛的应用。
        \item [Remarque.]事实:当$A$和$B$互素时,$M$可逆。
        \item [Remarque.]当$d=3,e=2$时,$M$如下
        \[
            \begin{pmatrix}
                a_0 &   0 & b_0 &   0 &  0\\
                a_1 & a_0 & b_1 & b_0 &  0\\
                a_2 & a_1 & b_2 & b_1 &  b_0\\
                a_3 & a_2 & 0   & b_2 &  b_1\\
                0   & a_3 & 0   & 0   &  b_2\\
            \end{pmatrix}.
        \]
    \end{enumerate}
    \begin{proof}1.
    \[
        M^T=N=
        \begin{pmatrix}
            a_0 & a_1 & \cdots &    a_d &     &        &     \\
                & a_0 &    a_1 & \cdots & a_d &        &     \\
                &     & \ddots & \ddots &     & \ddots &     \\
                &     &        &    a_0 & a_1 & \cdots & a_d \\
            b_0 & b_1 & \cdots &    b_e &     &        &     \\
                & b_0 &    b_1 & \cdots & b_e &        &     \\
                &     & \ddots & \ddots &     & \ddots &     \\
                &     &        &    b_0 & a_1 & \cdots & b_e \\
        \end{pmatrix}
    \]
    \[
        \mbox{记}
        N(B)=
        \begin{pmatrix}
            a_0 & a_1 & \cdots &    a_d &       &        &     \\
                & a_0 &    a_1 & \cdots & a_d   &        &     \\
                &     & \ddots & \ddots &       & \ddots &     \\
                &     &        &    a_0 & a_1   & \cdots & a_d \\
            b_0 & b_1 & \cdots &  b_e-B &       &        &     \\
                & b_0 &    b_1 & \cdots & b_e-B &        &     \\
                &     & \ddots & \ddots &       & \ddots &     \\
                &     &        &    b_0 & a_1   & \cdots & b_e-B \\
        \end{pmatrix}.
    \]
    记$x^i=x_i,l=d+e-1$,考虑关于$x_{l},x_{l-1},\ldots,x_0$的线性方程组:
    \begin{align*}
        &(1)\begin{cases}
            \ a_0 x_l \quad +a_1 x_{l-1} +\cdots +a_d x_{e-1}&=0\\
            \ a_0 x_{l-1} +a_1 x_{l-2} +\cdots +a_d x_{e-2}&=0\\
            \ \qquad \qquad \qquad \vdots &\\
            \ a_0 x_d \quad +a_1 x_{d-1} +\cdots +a_d x_0&=0\\
        \end{cases}
        \\
        &(2)\begin{cases}
            \ b_0 x_l \quad +b_1 x_{l-1} +\cdots +(b_e-B)x_{d-1}&=0\\
            \ b_0 x_{l-1} +b_1 x_{l-2} +\cdots +(b_e-B)x_{d-2}&=0\\
            \ \qquad \qquad \qquad \vdots &\\
            \ b_0 x_e \quad +b_1 x_{e-1} +\cdots +(b_e-B)x_0&=0
        \end{cases}
    \end{align*}
    $(1)$对于$A$的解$\lambda_1,\ldots,\lambda_d$成立;$(2)$对于$B$为恒等式.

    又$x_0=1$,$A$的解$\lambda_1,\ldots,\lambda_d$对应$N(B)X=0$的$d$组非平凡解,即$rank\ N(B)<d+e,\det N(B)=0$:
    \[
        \det N(B)=c_0 B^d+ c_1 B^{d-1}+ \cdots+ c_d=0
    \]
    $\det N(B)$作为$B$的$d$次多项式,它的$d$个根为
    \[B(\lambda_1),B(\lambda_2),\ldots,B(\lambda_d)\]
    在上述$\det N(B)$的展开式中,显然有
    \[c_0={(-1)}^d {a_0}^e,res(A,B)=\det N(0)=c_d\]
    由根和系数的关系,得:
    \[\prod\limits_{i=1}^d B(\lambda_i)={(-1)}^d \frac{c_d}{c_0}\]
    即:
    \[res(A,B)={a_0}^e \prod\limits_{i=1}^d B(\lambda_i)\]
    又$B=b_0\prod\limits_{j=1}^e (x-\mu_j)$:
    \[
        res(A,B)
        ={a_0}^e {b_0}^d \prod\limits_{i=1}^d \prod\limits_{j=1}^e (\lambda_i-\mu_j)
        ={a_0}^e {b_0}^d \prod\limits_{1\leq i\leq d \atop 1\leq j\leq e}(\lambda_i -\mu_j)
    \]
    \end{proof}
    \begin{proof}2.
    记$f(x)=m_0 x^e + m_1 x^{e-1} + m_e,g(x)=n_0 x^d + n_1 x^{d-1} + n_d$.
    解:
    \[
        f(x)A+g(x)B=1
        \Leftrightarrow
        \begin{pmatrix}
            a_0    &   0    & \cdots &      0 &    b_0 &      0 & \cdots & 0\\
            a_1    & a_0    & \cdots &      0 &    b_1 &    b_0 & \cdots & 0\\
            a_2    & a_1    & \ddots &      0 &    b_2 &    b_1 & \ddots & 0\\
            \vdots & \vdots & \ddots &    a_0 & \vdots & \vdots & \ddots & b_0\\
            a_d    & a_{d-1}& \cdots & \vdots &    b_e & b_{e-1}& \cdots & \vdots\\
            0      & a_d    & \ddots & \vdots &      0 &    b_e & \ddots & \vdots\\
            \vdots & \vdots & \ddots & a_{d-1}& \vdots & \vdots & \ddots & b_{e-1}\\
            0      &   0    & \cdots &    a_d &      0 &      0 & \cdots & b_e
        \end{pmatrix}
        \begin{pmatrix}
            m_0\\
            m_1\\
            \vdots\\
            m_e\\
            n_0\\
            n_1\\
            \vdots\\
            n_d
        \end{pmatrix}
        =
        MX=
        \begin{pmatrix}
            0\\
            0\\
            \vdots\\
            0\\
            1
        \end{pmatrix}
    \]
    \begin{align*}
        f(x)A+g(x)B=1\mbox{有解}&\Leftrightarrow f(x),g(x)\mbox{互素}\\
        MX={(0,0,\ldots,0,1)}^T \mbox{有解}&\Leftrightarrow \det(M)\neq 0\\
        f(x),g(x)\mbox{互素}&\Leftrightarrow \det(M)\neq 0
    \end{align*}
    即:
    \[\det(M)=0 \Leftrightarrow \exists 1\leq i \leq d, 1\leq j\leq e,\lambda_i=\mu_j \]
    可得:
    \[\det(M)=\xi \prod\limits_{1\leq i\leq d \atop 1\leq j\leq e}(\lambda_i -\mu_j)(\xi\mbox{为系数})\]
    取$\lambda_1=\cdots=\lambda_d=0$,则$a_0\neq 0,a_1=\cdots=a_d=0$
    \[
        \det(M)=
        \begin{vmatrix}
            a_0    &   0     & \cdots &      0 &    b_0 &      0 & \cdots & 0\\
            0      & a_0     & \cdots &      0 &    b_1 &    b_0 & \cdots & 0\\
            0      & 0       & \ddots &      0 &    b_2 &    b_1 & \ddots & 0\\
            \vdots & \vdots  & \ddots &    a_0 & \vdots & \vdots & \ddots & b_0\\
            0      & 0       & \cdots & \vdots &    b_e & b_{e-1}& \cdots & \vdots\\
            0      & 0       & \ddots & \vdots &      0 &    b_e & \ddots & \vdots\\
            \vdots & \vdots  & \ddots &      0 & \vdots & \vdots & \ddots & b_{e-1}\\
            0      &   0     & \cdots &      0 &      0 &      0 & \cdots & b_e
        \end{vmatrix}
        ={a_0}^e {b_e}^d={a_0}^e {({(-1)}^e b_0\cdot \mu_1 \cdot \mu_2 \cdots \mu_e)}^d
    \]
    又
    \begin{align*}
        \det(M)&={a_0}^e {({(-1)}^e b_0\cdot \mu_1 \cdot \mu_2 \cdots \mu_e)}^d\\
        &={a_0}^e {b_0}^d {(-1)}^{de} {(\mu_1 \cdot \mu_2 \cdots \mu_e)}^d\\
        &={a_0}^e {b_0}^d \prod\limits_{1\leq i\leq d \atop 1\leq j\leq e}{\lambda_i-\mu_j}
    \end{align*}
    可推出$\xi={a_0}^e {b_0}^d$,即证$res(A,B)={a_0}^e {b_0}^d \prod\limits_{1\leq i\leq d \atop 1\leq j\leq e}(\lambda_i -\mu_j)$.
    \end{proof}
\end{enumerate}
\section{周四}
\subsection{习题四}
\begin{enumerate}
    \item [21.]求以下排列的逆序数,并指出其奇偶性:
    \begin{enumerate}
        \item [(1)] (6,8,1,4,7,5,3,2,9) \quad 逆序数19,奇
        \item [(2)] (6,4,2,1,9,7,3,5,8) \quad 逆序数15,奇
        \item [(3)] (7,5,2,3,9,8,1,6,4) \quad 逆序数20,偶
    \end{enumerate}
    \item [23.]
    \begin{enumerate}
        \item [(4)]
        \[
            \begin{vmatrix}
                    &         &     & A_1\\
                    &         & A_2 &    \\
                    & \iddots &          \\
                A_k &         &     &     
            \end{vmatrix}
            \xrightarrow{n_1\sum\limits_{i=2}^{k}n_i\mbox{次相邻对调}}
            {(-1)}^{\sum\limits_{i=2}^{k}n_1n_i}
            \begin{vmatrix}
                A_1 &     &         &     \\
                    &     &         & A_2 \\
                    &     & \iddots &     \\
                    & A_k &         &     
            \end{vmatrix}
        \]
        即:
        \[
            \begin{vmatrix}
                &         &     & A_1\\
                &         & A_2 &    \\
                & \iddots &          \\
            A_k &         &     &     
            \end{vmatrix}
            =
            {(-1)}^{\sum\limits_{i<j} n_i n_j}
            \begin{vmatrix}
                A_1 &     &        &     \\
                    & A_2 &        &     \\
                    &     & \ddots &     \\
                    &     &        & A_k 
            \end{vmatrix}
            =
            {(-1)}^{\sum\limits_{i<j} n_i n_j}
            \det(A_1)\cdots\det(A_k)
        \]
        \item [(6)]
        \begin{align*}
            \begin{vmatrix}
                1+a_1 &     1 & \cdots&      1\\
                    1 & 1+a_2 & \ddots& \vdots\\  
                \vdots& \ddots& \ddots&      1\\
                    1 & \cdots&     1 &  1+a_n\\
                \end{vmatrix}
                &=
                \begin{vmatrix}
                        1 &     1 &     1 & \cdots&      1\\
                        0 & 1+a_1 &     1 & \cdots&      1\\
                        0 &     1 & 1+a_2 & \ddots& \vdots\\  
                        0 & \vdots& \ddots& \ddots&      1\\
                        0 &     1 & \cdots&     1 &  1+a_n\\
                \end{vmatrix}
                =
                \begin{vmatrix}
                    1 &     1 &     1 & \cdots&      1\\
                   -1 &   a_1 &     0 & \cdots&      0\\
                   -1 &     0 &   a_2 & \ddots& \vdots\\  
                   -1 & \vdots& \ddots& \ddots&      0\\
                   -1 &     0 & \cdots&     0 &    a_n\\
                \end{vmatrix}\\
                &=
                \begin{vmatrix}
                    1+\sum\limits_{i=1}^n \frac{1}{a_i} &     0 &     0 & \cdots&      0\\
                   -1 &   a_1 &     0 & \cdots&      0\\
                   -1 &     0 &   a_2 & \ddots& \vdots\\  
                   -1 & \vdots& \ddots& \ddots&      0\\
                   -1 &     0 & \cdots&     0 &    a_n\\
                \end{vmatrix}
                =
                \left(1+\sum\limits_{i=1}^n \frac{1}{a_i}\right)\prod\limits_{j=1}^n a_j.
        \end{align*}
        \item [(8)]
        \begin{enumerate}
            \item [(a)]$n=1$ \[\begin{vmatrix} a_1-b_1 \end{vmatrix}=a_1-b_1.\]
            \item [(b)]$n=2$ 
            \[
                \begin{vmatrix}
                    a_1-b_1 & a_1-b_2\\
                    a_2-b_1 & a_2-b_2
                \end{vmatrix}
                =(a_1-b_1)(a_2-b_2)-(a_1-b_2)(a_2-b_1).
            \]
            \item [(c)]$n\geq 3$
            \[
                \begin{vmatrix}
                    a_1-b_1 & a_1-b_2 & \cdots & a_1-b_n\\
                    a_2-b_1 & a_2-b_2 & \cdots & a_2-b_n\\
                    \vdots  & \vdots  &        & \vdots \\
                    a_n-b_1 & a_n-b_2 & \cdots & a_n-b_n
                \end{vmatrix}
                =
                \begin{vmatrix}
                    a_1-b_1 & b_1-b_2 & \cdots & b_1-b_n\\
                    a_2-b_1 & b_1-b_2 & \cdots & b_1-b_n\\
                    \vdots  & \vdots  &        & \vdots \\
                    a_n-b_1 & b_1-b_2 & \cdots & b_1-b_n
                \end{vmatrix}
                =0.
            \]
        \end{enumerate}
    \end{enumerate}
\end{enumerate}
\subsection{补充题}
\begin{enumerate}
    \item [1.]确定$i,j$,使得$(1245i6j97)$分别为奇排列、偶排列。
    \begin{enumerate}
        \item [(1)]$i=3,j=8$,逆序数为$4$,偶排列;
        \item [(2)]$i=8,j=3$,逆序数为$7$,奇排列.
    \end{enumerate}
    \item [2.]将$\lambda$作为变量,$a_{ij}(1\leq i,j\leq n)$作为常数,则
    \[
        f_{n}(\lambda)=
        \begin{vmatrix}
            \lambda-a_{11} & -a_{12} & \cdots & -a_{1n}\\
            -a_{21} & \lambda-a_{22} & \cdots & -a_{2n}\\
            \vdots  & \vdots  & \ddots & \vdots \\
            -a_{n1} & -a_{n2} & \cdots & \lambda-a_{nn}\\
        \end{vmatrix}
    \]
    为$\lambda$的多项式,求这个多项式的$n$次项和$n-1$次项。
    \[
        \mbox{记}f_{n}(\lambda)=
        (\lambda-a_{nn})f_{n-1}(\lambda)
        \begin{comment}
            \begin{vmatrix}
            \lambda-a_{11} & -a_{12} & \cdots & -a_{1(n-1)}\\
            -a_{32} & \lambda-a_{33} & \cdots & -a_{2(n-1)}\\
            \vdots  & \vdots  & \ddots & \vdots \\
            -a_{(n-1)2} & -a_{(n-1)3} & \cdots & \lambda-a_{(n-1)(n-1)}\\
            \end{vmatrix}
        \end{comment}
        +a_{12}
        \begin{vmatrix}
            -a_{21} & -a_{23} & \cdots & -a_{2n}\\
            -a_{31} & \lambda-a_{33} & \cdots & -a_{3n}\\
            \vdots  & \vdots  & \ddots & \vdots \\
            -a_{n1} & -a_{n3} & \cdots & \lambda-a_{nn}\\
        \end{vmatrix}
        +\cdots.
    \]
    显然,后$n-1$个项中$\lambda$的次数均不超过$n-2$,仅第一项中含$n$次项和$n-1$次项,即在$f_{n-1}(\lambda)$中找$n-1$次项和$n-2$次项.

    依此类推,可得$f(x)$的$n$次项和$n-1$次项均在$\prod\limits_{i=1}^n (\lambda-a_{ii})$里.
    \begin{align*}
        n\mbox{次项}&:\lambda^{n};\\
        n-1\mbox{次项}&:-\sum\limits_{i=1}^{n} a_{ii} \lambda^{n-1}.
    \end{align*}
\end{enumerate}
\end{document}